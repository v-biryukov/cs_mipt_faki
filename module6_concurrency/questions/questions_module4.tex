\documentclass{article}
\usepackage[utf8x]{inputenc}
\usepackage{ucs}
\usepackage{amsmath} 
\usepackage{amsfonts}
\usepackage{upgreek}
\usepackage[english,russian]{babel}
\usepackage{graphicx}
\usepackage{float}
\usepackage{textcomp}
\usepackage{hyperref}
\usepackage{geometry}
  \geometry{left=2cm}
  \geometry{right=1.5cm}
  \geometry{top=1cm}
  \geometry{bottom=2cm}
\usepackage{tikz}
\usepackage{ccaption}
\usepackage{mathrsfs}
\usepackage[shortlabels]{enumitem}
\usepackage{multicol}
\usepackage{listings}
\lstset{
  language=C,                % choose the language of the code
  basicstyle=\linespread{1.1}\ttfamily,
  columns=fixed,
  fontadjust=true,
  basewidth=0.5em,
  keywordstyle=\color{blue}\bfseries,
  commentstyle=\color{gray},
  stringstyle=\ttfamily\color{orange!50!black},
  showstringspaces=false,
  %numbers=false,                   % where to put the line-numbers
  numbersep=5pt,
  numberstyle=\tiny\color{black},
  numberfirstline=true,
  stepnumber=1,                   % the step between two line-numbers.        
  numbersep=10pt,                  % how far the line-numbers are from the code
  backgroundcolor=\color{white},  % choose the background color. You must add \usepackage{color}
  showstringspaces=false,         % underline spaces within strings
  captionpos=b,                   % sets the caption-position to bottom
  breaklines=true,                % sets automatic line breaking
  breakatwhitespace=true,         % sets if automatic breaks should only happen at whitespace
  xleftmargin=.2in,
  extendedchars=\true,
  keepspaces = true,
}
\lstset{literate=%
   *{0}{{{\color{red!20!violet}0}}}1
    {1}{{{\color{red!20!violet}1}}}1
    {2}{{{\color{red!20!violet}2}}}1
    {3}{{{\color{red!20!violet}3}}}1
    {4}{{{\color{red!20!violet}4}}}1
    {5}{{{\color{red!20!violet}5}}}1
    {6}{{{\color{red!20!violet}6}}}1
    {7}{{{\color{red!20!violet}7}}}1
    {8}{{{\color{red!20!violet}8}}}1
    {9}{{{\color{red!20!violet}9}}}1
}


\begin{document}
\pagenumbering{gobble}

\section*{Модуль "Утилиты". Вопросы.}
\begin{enumerate}

\item \textbf{Сборка}
\begin{enumerate}[a.]
\item \textbf{Раздельная компиляция}\\
Что такое файл исходного кода и исполняемый файл? Этап сборки программы: препроцессинг, ассемблирование, компиляция и линковка. Что такое заголовочные файлы (header-файлы)? Что делает директива препроцессора \texttt{\#include}? Что такое единица трансляции? Компиляция программы с помощью \texttt{g++}. Опции компиляции \texttt{-E}, \texttt{-S} и \texttt{-c}. Что такое раздельная компиляция и в чём её преемущества?


\item \textbf{Библиотеки}\\
Что такое библиотека? Виды библиотек: header-only библиотеки, open-source библиотеки, статические библиотеки, динамические библиотеки. В чём различия между этими видами библиотек? В чём преимущества и недостатки каждого из видов библиотек? Как подключить библиотеки к своему проекту? 

\item \textbf{Статические библиотеки}\\
Как создать статическую библиотеку? Как подключить статическую библиотеку? Опции компилятора \texttt{-I}, \texttt{-L} и \texttt{-l}. Характерные расширения файлов статических библиотек на Linux и Windows. 

\item \textbf{Динамические библиотеки}\\
В чём главная разница между статическими и динамическими библиотеками? Как создать динамическую библиотеку? Как подключить динамическую библиотеку? Характерные расширения файлов динамических библиотек на Linux и Windows. 

\item \textbf{Опции компилятора \texttt{g++}}
\begin{itemize} 
\item Опции для указания стандарта языка, например \texttt{-std=c++20}
\item Опции для включения/отключения предупреждений: \texttt{-Wall}, \texttt{-Wextra}, \texttt{-Werror}.
\item Опция для указания директорий заголовочных файлов, необходимых для компиляции \texttt{-I}
\item Опция для указания директорий библиотек, необходимых для компиляции \texttt{-L}
\item Опция для указания названий библиотек, необходимых для компиляции \texttt{-l}
\item Опция для включения возможности проведения дебага: \texttt{-g}
\item Опции для включения оптимизаций: \texttt{-O0}, \texttt{-O1}, \texttt{-O2}, \texttt{-O3}, \texttt{-Os}
\item Опция \texttt{-DNDEBUG}
\item Опция \texttt{-D} для задания \texttt{\#define}-макросов. Как ёё использовать? Пример использования данной опции.
\end{itemize}
\end{enumerate}







\item \textbf{CMake как система сборки}
\begin{enumerate}[a.]

\item \textbf{Основы CMake}\\
Что такое Cmake и для чего он нужен? Основы работы с CMake. Структура CMake-проекта. Файл CMakeListis.txt.
Как скомпилировать проект с помощью CMake? Что делают следующие команды CMake:
\begin{itemize}
\item \texttt{cmake\_minimum\_required}
\item \texttt{project}
\item \texttt{add\_executable}
\item \texttt{message}
\end{itemize}

Как собрать проект с использованием CMake? Генерация файлов проекта для данной среды. Выбор генератора.
Опции программы \texttt{cmake}: \texttt{-S}, \texttt{-B}, \texttt{-G}, \texttt{-{}-build}.


\item \textbf{Таргеты}\\
Что такое таргет (target)? Что делают следующие команды CMake:
\begin{itemize}
\item \texttt{add\_executable}
\item \texttt{add\_library} и её опции \texttt{STATIC} и \texttt{SHARED}
\item \texttt{target\_link\_libraries} (если аргумент является таргетом)
\end{itemize}


\item \textbf{Свойства таргетов}\\
Что делают следующие команды CMake:
\begin{itemize}
\item \texttt{target\_include\_directories}
\item \texttt{target\_link\_directories}
\item \texttt{target\_link\_libraries} (если аргумент не является таргетом)
\item \texttt{target\_compile\_features}
\item \texttt{target\_compile\_definitions}
\item \texttt{target\_compile\_options}
\end{itemize}


\item \textbf{Типы зависимостей между таргетам}\\
Типы связей между двумя таргетам \texttt{PRIVATE}, \texttt{PUBLIC} и \texttt{INTERFACE}.
Типы связей между таргетом и его свойством \texttt{PRIVATE}, \texttt{PUBLIC} и \texttt{INTERFACE}.
В чём отличия между этими типами зависимостей? Зачем нужно указывать тип для каждой связи? 
Примеры ситуаций когда нужно использовать ту или иную связь.


\item \textbf{Простые переменные CMake}\\
Простые переменные CMake. Какие бывают типы у переменных языка CMake? Как создать простую переменную в CMake? Команда \texttt{set}. Как напечатать значение переменной на экран? Основные стандартные переменные:
\begin{multicols}{2}
\begin{itemize}
\item \texttt{CXX\_STANDARD}
\item \texttt{CMAKE\_CXX\_COMPILER}
\item \texttt{CMAKE\_SOURCE\_DIR}
\item \texttt{CMAKE\_BUILD\_DIR}
\item \texttt{PROJECT\_SOURCE\_DIR}
\item \texttt{PROJECT\_BUILD\_DIR}
\item \texttt{<имя проекта>\_SOURCE\_DIR}
\item \texttt{<имя проекта>\_BUILD\_DIR}
\item \texttt{BUILD\_SHARED\_LIBS}
\item \texttt{WIN32}, \texttt{LINUX}, \texttt{APPLE}, \texttt{MSVC}, \texttt{MINGW}
\end{itemize}
\end{multicols}

\item \textbf{Поддиректории}\\
Как добавить новую поддиректорию в CMake проект? Команда \texttt{add\_subdirectory}. Что происходит при выполнении этой команды?
Область видимости переменных. Видны ли переменные, созданные в родительской Cmake-директории, в поддиректории? Видны ли переменные, созданные в поддиректории, в родительской Cmake-директории? Опция \texttt{PARENT\_SCOPE} команды \texttt{set}. Переменные:
\begin{itemize}
\item \texttt{CMAKE\_CURRENT\_SOURCE\_DIR}
\item \texttt{CMAKE\_CURRENT\_BUILD\_DIR}
\end{itemize}

\end{enumerate}





\item \textbf{CMake как язык программирования}
\begin{enumerate}[a.]

\item \textbf{Переменные}\\
Какие бывают типы у переменных языка CMake? Как создать простую переменную в CMake? Команда \texttt{set}. Как получить значение переменной по её названию?

\item \textbf{Условная команда \texttt{if}}\\
Как пользоваться командой \texttt{if} и сопутствующими командами в языке CMake? Какие строки команда \texttt{if} воспринимает как истинные, а какие как ложные? Использование переменных как аргуметы команды \texttt{if}. Логические операторы \texttt{AND}, \texttt{OR}, \texttt{NOT}. Сравнение чисел: \texttt{EQUAL},  \texttt{LESS},  \texttt{GREATER}. Сравнение строк на равенство: \texttt{STREQUAL}. Проверка на то, что существует файл: \texttt{EXISTS}. Проверка, существует ли переменная и данным именем: \texttt{DEFINED}.

\item \textbf{Списки}\\
Что представляет собой список в языке CMake? Как создать список? Как работать со списком? Передача списка в функцию. Команда \texttt{list}. Опции этой команды: \texttt{LENGTH}, \texttt{GET}, \texttt{FIND}, \texttt{APPEND}, \texttt{SORT}. 

\item \textbf{Циклы}\\
Команда \texttt{while}. Команда \texttt{foreach}. Опции команды \texttt{foreach}: \texttt{RANGE} и \texttt{IN LISTS}. Итерирование по списку с помощью команды \texttt{foreach}.

\item \textbf{Функции}\\
Функции в языке CMake. Как создать функцию с помощью команды \texttt{function}? Как передавать в функцию? Переменные \texttt{ARGC}, \texttt{ARGV}, \texttt{ARGN}. Как возвращать из функции. Опция \texttt{PARENT\_SCOPE} команды \texttt{set}. Команда \texttt{return} с опцией \texttt{PROPAGATE}. Области видимости функций. Команда \texttt{cmake\_parse\_arguments}.

\item \textbf{Манипуляции со строками}\\
Команда \texttt{string} и её опции:
\begin{multicols}{3}
\begin{itemize}
\item \texttt{FIND}
\item \texttt{REPLACE}
\item \texttt{APPEND}
\item \texttt{JOIN}
\item \texttt{TOLOWER}
\item \texttt{TOUPPER}
\item \texttt{LENGTH}
\item \texttt{SUBSTRING}
\item \texttt{COMPARE}
\end{itemize}
\end{multicols}

\newpage
\item \textbf{Файлы}\\
Команда \texttt{file} и её опции:
\begin{multicols}{3}
\begin{itemize}
\item \texttt{READ}
\item \texttt{STRINGS}
\item \texttt{WRITE}
\item \texttt{MAKE\_DIRECTORY}
\item \texttt{REMOVE}
\item \texttt{RENAME}
\item \texttt{COPY}
\item \texttt{SIZE}
\item \texttt{CHMOD}
\item \texttt{REAL\_PATH}
\item \texttt{DOWNLOAD}
\item \texttt{GLOB}
\end{itemize}
\end{multicols}
Является ли хорошей идеей использование команды \texttt{file} с опцией \texttt{GLOB}, чтобы найти названия всех файлов исходного кода некоторого таргета?

\item \textbf{Модули}\\
Что представляет собой модуль в языке CMake. Подключение модулей. Команда \texttt{include}. В каких папках ищутся модули? Переменная \texttt{CMAKE\_MODULE\_PATH}. Область видимости переменных. Переменные \texttt{CMAKE\_CURRENT\_LIST\_DIR} и \texttt{CMAKE\_CURRENT\_LIST\_FILE}. Чем команда \texttt{include} отличается от команды \texttt{add\_subdirectory}?
Команда \texttt{include\_guard}.

\end{enumerate}






\item \textbf{CMake - дополнительные возможности}

\begin{enumerate}[a.]
\item \textbf{Переменные среды}\\
Что такое переменные среды? Как пользоваться переменными среды из CMake?

\item \textbf{Кэшированные переменные CMake}\\
Чем кэшированные переменные отличаются от обычных переменных? Создание кэшированных переменных с помощью команды \texttt{set}. Поле \texttt{type} при создании кешированной переменной. Как изменить уже созданную ранее кэшированную переменную? Опция \texttt{FORCE} команды \texttt{set}. Задание кешированных переменных в командной строке (опция \texttt{-D}). Файл \texttt{CMakeCache.txt}.

\item \textbf{Свойства}\\
Что такое свойства в языке CMake? В чём отличие свойств от переменных? Какие объекты могут иметь свойства?
Команды \texttt{get\_property} и \texttt{set\_property} для получения и изменения свойств различных объектов.
Cвойства директорий:
\begin{itemize}
\item \texttt{VARIABLES}
\item \texttt{CACHE\_VARIABLES}
\item \texttt{SUBDIRECTORIES}
\item \texttt{PARENT\_DIRECTORY}
\item \texttt{BUILDSYSTEM\_TARGETS}
\item \texttt{IMPORTED\_TARGETS}
\end{itemize}

Cвойства таргетов:
\begin{multicols}{2}
\begin{itemize}
\item \texttt{TYPE}
\item \texttt{OUTPUT\_NAME}
\item \texttt{SOURCES}
\item \texttt{INCLUDE\_DIRECTORIES}
\item \texttt{COMPILE\_DEFINITIONS}
\item \texttt{COMPILE\_OPTIONS}
\item \texttt{INTERFACE\_INCLUDE\_DIRECTORIES}
\item \texttt{INTERFACE\_COMPILE\_DEFINITIONS}
\item \texttt{INTERFACE\_COMPILE\_OPTIONS}
\item \texttt{LINK\_DIRECTORIES}
\item \texttt{LINK\_LIBRARIES}
\item \texttt{LINK\_OPTIONS}
\item \texttt{INTERFACE\_LINK\_DIRECTORIES}
\item \texttt{INTERFACE\_LINK\_LIBRARIES}
\item \texttt{INTERFACE\_LINK\_OPTIONS}
\end{itemize}
\end{multicols}

\item \textbf{Тип сборки}\\
Что такое тип сборки (также известный как \textit{тип конфигурации} или просто \textit{конфиг})?\\
Типы сборки по умолчанию:
\begin{itemize}
\item \texttt{Release}
\item \texttt{Debug}
\item \texttt{RelWithDebInfo}
\item \texttt{MinSizeRel}
\item \texttt{"{}"} (пустой)
\end{itemize}
Чем отличаются эти типы сборки? Какие опции компилятора использует каждый из этих типов сборки?
Одноконфигурационные (single-config) и мультиконфигурационные (multi-config) генераторы. Как установить тип сборки при использовании одноконфигурационного генератора? Переменная \texttt{CMAKE\_BUILD\_TYPE}. Как установить тип сборки при использовании мультиконфигурационного генератора? Переменные \texttt{CMAKE\_CONFIGURATION\_TYPES} и \texttt{GENERATOR\_IS\_MULTI\_CONFIG}. Как узнать тип сборки при использовании мультиконфигурационного генератора?


\item \textbf{Генераторные выражения}\\
Стадия конфигурации и стадия генерации. Что такое генераторные выражения (generator expressions)? Когда их нужно использовать?
Приведите пример команд, которые поддерживают генераторные выражения и команд, которые их не поддерживают.
Синтаксис генераторных выражений. Что делают следующие генераторные выражения:
\begin{verbatim}
0, 1, BOOL, AND, OR, NOT, IF, STREQUAL, CONFIG, TARGET_PROPERTY, TARGET_FILE
\end{verbatim}

\end{enumerate}


\item \textbf{Git - локальный репозиторий}

\begin{enumerate}[a.]
\item \textbf{Основы}\\
Что такое система контроля версий? Централизованные и распределённые системы контроля версий. Снимок состояния. Репозиторий. Локальный и удалённый репозитории.
\item \textbf{Настройка git}\\
Команда \texttt{git config} и её использование для настройки git. Что делают следующие команды:
\begin{lstlisting}
git config user.name "Ivan Ivanov"
git config user.email ivan.ivanov@mail.ru
git config core.editor vim
git config core.autocrlf true
\end{lstlisting}
Опции команды \texttt{git config}: \texttt{-{}-local}, \texttt{-{}-global} и \texttt{-{}-list}.
\item \textbf{Создание нового репозиттория}\\
Создание нового пустого репозитория с помощью команды \texttt{git init}. Клонирование существующего репозитория с помощью команды \texttt{git clone}.

\item \textbf{Добавление новых коммитов}\\
Области хранения файлов в системе git:
\begin{itemize}
\item рабочая папка
\item индекс
\item локальный репозиторий
\item удалённый репозиторий
\end{itemize}

Типы файлов в системе git:
\begin{itemize}
\item неотслеживаемые
\item игнорируемые
\item индексированные
\item изменённые
\item зафиксированные (закомиченные)
\end{itemize}

Как создать неотслеживаемый файл?
Добавление неотслеживаемых файлов в индекс с помощью команды \texttt{git add}. 
Как изменить/удалить файлы, находящиеся в индексе?
Добавление индексированных файлов в локальный репозиторий с помощью команды \texttt{git commit}.
Что такое коммит?
Опции команды \texttt{git commit}: \texttt{-m} и \texttt{-a}.
Игнорируемые файлы. Файл \texttt{.gitignore}.

\item \textbf{Просмотр информации}\\
Просмотр состояния рабочей папки и индекса с помощью команды \texttt{git status}.
Команда \texttt{git diff} для просмотра разницы в файлах рабочей папки и файлах индекса.
Команда \texttt{git diff -{}-staged} для просмотра разницы в файлах индекса и файлах текущего коммита репозитория.
Просмотр истории коммитов с помощью команды \texttt{git log}. Опции команды \texttt{git log}: \texttt{-{}-oneline}, \texttt{-{}-graph}, \texttt{-{}-since}, \texttt{-S}. Хеш коммита. Сокращённый хеш коммита. Команда \texttt{git diff} для просмотра разницы в файлах двух коммитов.

\item \textbf{Очистка и отмена}\\
Удаление всех неотслеживаемых файлов с помощью \texttt{git clean -fd}.
Удаление всех игнорируемых файлов с помощью команды \texttt{git clean -fdX}.
Отмена изменений в рабочей директории с помощью \texttt{git restore}. 
Отмена изменений в индексе с помощью \texttt{git restore -{}-staged}.
Правка последнего коммита с помощью команды \texttt{git commit -{}-amend}. 
Отмена зафиксированных изменений, с помощью \texttt{git reset}.
Чем отличаются следующие команды:
\begin{itemize}
\item \texttt{git reset -{}-hard}
\item \texttt{git reset -{}-mixed}
\item \texttt{git reset -{}-soft}
\end{itemize}
Можно ли восстановить данные, если вы случайно сделали \texttt{git reset -{}-hard}? 
Как полностью удалить коммит из репозитория?

\item \textbf{Ветки}\\
Что такое ветки в системе git? Создание веток с помощью команды \texttt{git branch}. Ветка \texttt{master} (\texttt{main}). Удаление веток с помощью команды \texttt{git branch -d}. Переименование веток командой \texttt{git branch -m}. Перенос ветки на другой коммит с помощью команды \texttt{git branch -f}. Переход на другую ветку с помощью команды \texttt{git switch}. Создание новой ветки с переходом на неё с помощью \texttt{git switch -c}. Указатель \texttt{HEAD}. Переход на произвольный коммит с помощью \texttt{git switch}. Состояние "отделённой головы" (detached HEAD). Как выйти из состояния отделённой головы?
Использование символов \texttt{$\sim$} и \texttt{\textasciicircum} для навигации по коммитам репозитория.

\item \textbf{Слияние веток}\\
Слияние веток с помощью команды \texttt{git merge}. Слияние веток с помощью "перемотки" (fast-forwarding). Конфликты слияния. Разрешение конфликтов слияния и команда \texttt{git merge -{}-continue}. Отмена конфликтного слияния, команда \texttt{git merge -{}-abort}. Как отменить уже сделанное слияние? Опции \texttt{-{}-no-commit} и \texttt{-{}-no-ff} команды \texttt{git merge}.

\item \textbf{Копирование коммитов}\\
Копирование коммитов с помощью \texttt{git cherry-pick}. Конфликты при копировании коммитов. Разрешение конфликтов при копировании и команда \texttt{git cherry-pick -{}-continue}. Отмена конфликтного копирования, команда \texttt{git cherry-pick -{}-abort}.

\item \textbf{Перебазирование веток}\\
Перебазирование веток с помощью \texttt{git rebase}. Отличие слияния от перебазирования. Конфликты при перебазировании. Разрешение конфликтов при перебазировании и команда \texttt{git rebase -{}-continue}. Отмена конфликтного перебазирования, команда \texttt{git rebase -{}-abort}.
Интерактивное перебазирование. Действия при интерактивном перебазировании: \texttt{pick}, \texttt{reword}, \texttt{edit}, \texttt{squash}, \texttt{drop}.
\end{enumerate}


\item \textbf{Git - удалённые репозитории}
\begin{enumerate}[a.]
\item \textbf{Удалённый репозиторий}\\
Просмотр удалённых репозиториев. Команда \texttt{git remote -v}. Удалённый репозиторий \texttt{origin}. Добавление ссылки на удалённый репозиторий с помощью команды \texttt{git remote add}. 

\item \textbf{Удалённые ветки и ветки слежения}\\
Удалённые ветки. Просмотр удалённых веток с помощью команды \texttt{git branch -a}. Ветки слежения. Связывание локальной ветки с удалённой веткой. Команда \texttt{git push -u}. Просмотр локальных веток и их связей с удалёнными ветками с помощью команды \texttt{git branch -vv}. Удаление веток на удалённом сервере.

\item \textbf{Синхронизация с удалённым репозиторием}\\
Загрузка изменений из удалённого репозитория. Команда \texttt{git fetch}. Какую ветку обновляет команда \texttt{git fetch}? Загрузка изменений со слиянием с помощью команды \texttt{git pull}. Загрузка изменений с перебазированием с помощью команды \texttt{git pull -{}-rebase}. 
Отправка изменений на удалённый сервер с помощью команды \texttt{git push}.

\item \textbf{GitHub}\\
Хостинги git репозиториев. GitHub, GitLab, BitBucket. Доступ на GitHub по SSH. SSH-ключи.
Создание нового репозитория.
Добавление участников (collaborators).
Создание форков.
Пулл-реквесты.
\end{enumerate}


\newpage
\item \textbf{CMake - подключение сторонних библиотек}

\begin{enumerate}[a.]

\item \textbf{Подключение сторонней библиотеки с помощью \texttt{add\_subdirectory}}\\
Как подключить сторонний CMake проект к нашему CMake проекту с помощью команды \texttt{add\_subdirectory}.
В чём преемущества и недостатки такого подхода?
Различие между переменными \texttt{CMAKE\_SOURCE\_DIR} и \texttt{PROJECT\_SOURCE\_DIR}.

\item \textbf{Поиски файлов}\\
Поиск файла с помощью команды \texttt{find\_file}. Порядок поиска файла в системе.
Переменные вида \texttt{<packageName>\_ROOT}.
Переменные \texttt{CMAKE\_PREFIX\_PATH}, \texttt{CMAKE\_INCLUDE\_PATH} и \texttt{CMAKE\_FRAMEWORK\_PATH}.
Переменные среды \texttt{INCLUDE} и \texttt{PATH}.
Переменные \texttt{CMAKE\_SYSTEM\_PREFIX\_PATH}, \texttt{CMAKE\_SYSTEM\_INCLUDE\_PATH} и \texttt{CMAKE\_SYSTEM\_FRAMEWORK\_PATH}.
Опции \texttt{HINTS} и \texttt{PATHS} команды \texttt{find\_file}. Опция \texttt{PATH\_SUFFIXES}.
Опция \texttt{NO\_DEFAULT\_PATH}. Опция \texttt{REQUIRED}.
Команды \texttt{find\_path}, \texttt{find\_program} и \texttt{find\_library}. Чем эти команды отличаются от \texttt{find\_file}?
Переменные \texttt{CMAKE\_PROGRAM\_PATH} и \texttt{CMAKE\_LIBRARY\_PATH}.

\item \textbf{Поиск библиотек}\\
Команда \texttt{find\_package} для поиска установленных библиотек. Опция \texttt{REQUIRED}. Опция \texttt{COMPONENTS}.
Файл с именем вида \texttt{Find<packageName>.cmake} и алгоритм поиска такого файла.
Файл с именем вида \texttt{<packageName>Config.cmake} и алгоритм поиска такого файла.
Переменные вида \texttt{<packageName>\_DIR}.
Переменные вида \texttt{<packageName>\_FOUND}.  
Как узнать какие таргеты были импортированы в проект после вызова команды \texttt{find\_package}?
Опция \texttt{-{}-debug-find} программы \texttt{cmake}.

\item \textbf{FetchContent}\\
Модуль \texttt{FetchContent}.
Команда \texttt{FetchContent\_Declare} и её опции \texttt{GIT\_REPOSITORY}, \texttt{GIT\_TAG}, \texttt{URL}.
Команда \texttt{FetchContent\_MakeAvailable}.

\item \textbf{git submodule}\\
Что такое подмодули (submodules) в системе git? Добавление нового подмодуля с помощью команды \texttt{git submodule add}. 
Файл \texttt{.gitmodules}. 
Клонирование проекта вместе с подмодулями. Команда \texttt{git clone -{}-recurse-submodules}.
Инициализация подмодулей. Команда \texttt{git submodule update -{}-init}.
Загрузка изменений в подмодулях. Команда \texttt{git pull -{}-recurse-submodules}.
В чём преемущества и недостатки системы подмодулей git?
\end{enumerate}


\item \textbf{Тестирование}
\begin{enumerate}[a.]
\item \textbf{GoogleTest}\\
Юнит-тестирование. Библиотека GoogleTest. Написание тестов с помощью библиотеки GoogleTest.
Что такое тест? Что такое набор тестов? Макрос \texttt{TEST}. Утверждения. ASSERT-утверждения и EXPECT-утверждения. Чем они отличаются? Макросы \texttt{ASSERT\_TRUE}, \texttt{ASSERT\_FALSE}, \texttt{ASSERT\_EQ}, \texttt{ASSERT\_GT},\\ \texttt{ASSERT\_LT}, \texttt{ASSERT\_STREQ}, \texttt{ASSERT\_THROW}. Паттерн AAA. Фиксации (fixtures). Макрос \texttt{TEST\_F}. Класс \texttt{testing::Test}. Методы класса-фиксации \texttt{SetUp} и \texttt{TearDown}. Зачем нужно использовать фиксации? Запуск тестов. Функция \texttt{InitGoogleTest}. Макрос \texttt{RUN\_ALL\_TESTS}.
\item \textbf{CTest}\\
Что такое тест в CTest? Команда \texttt{enable\_testing}. Команда \texttt{add\_test}. Опции \texttt{NAME}, \texttt{COMMAND},\\ \texttt{WORKING\_DIRECTORY} команды \texttt{add\_test}.
Использование GoogleTest совместно с CTest. 
Запуск тестов с помощью программы ctest.
Опции программы \texttt{ctest}: \texttt{-C}, \texttt{-N}, \texttt{-{}-repeat}, \texttt{-{}-timeout}.

\end{enumerate}




\end{enumerate}

\end{document}