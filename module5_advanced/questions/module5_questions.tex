\documentclass{article}
\usepackage[utf8x]{inputenc}
\usepackage{ucs}
\usepackage{amsmath} 
\usepackage{amsfonts}
\usepackage{upgreek}
\usepackage[english,russian]{babel}
\usepackage{graphicx}
\usepackage{float}
\usepackage{textcomp}
\usepackage{hyperref}
\usepackage{geometry}
  \geometry{left=2cm}
  \geometry{right=1.5cm}
  \geometry{top=1cm}
  \geometry{bottom=2cm}
\usepackage{tikz}
\usepackage{ccaption}
\usepackage{mathrsfs}
\usepackage[shortlabels]{enumitem}
\usepackage{listings}
\lstset{
  language=C++,
  basicstyle=\linespread{1.1}\ttfamily,
  columns=fixed,
  fontadjust=true,
  basewidth=0.5em,
  keywordstyle=\color{blue}\bfseries,
  commentstyle=\color{gray},
  stringstyle=\ttfamily\color{orange!50!black},
  showstringspaces=false,
  %numbers=false,
  numbersep=5pt,
  numberstyle=\tiny\color{black},
  numberfirstline=true,
  stepnumber=1,     
  numbersep=10pt,
  backgroundcolor=\color{white},
  showstringspaces=false,
  captionpos=b,
  breaklines=true,
  breakatwhitespace=true,
  xleftmargin=.2in,
  extendedchars=\true,
  keepspaces = true,
}

\begin{document}
\pagenumbering{gobble}

\section*{Модуль 5. Дополнительные темы С++. Вопросы.}
\begin{enumerate}


\item \textbf{Паттерны проектирования с использованием шаблонов}
\begin{enumerate}[a.]
\item \textbf{Класс \texttt{any} из стандартной библиотеки}\\
Класс \texttt{any}. Функция \texttt{any\_cast}. 


\item \textbf{Класс \texttt{variant} из стандартной библиотеки}\\
Функции для работы с \texttt{variant}:
\begin{itemize}
\item \texttt{get}
\item \texttt{holds\_alternative}
\item \texttt{visit}
\end{itemize}
Для чего можно применять \texttt{variant}? Динамический полиморфизм при использовании класса \texttt{variant}.

\item \textbf{Type erasure}\\
Паттерн Type erasure (Стирание типа). Реализация своего класса \texttt{any} при помощи паттерна Type erasure.
\end{enumerate}

\item \textbf{Обработка ошибок}
\begin{enumerate}[a.]
\item \textbf{Методы обработки ошибок.}\\
Классификация ошибок. Ошибки времени компиляции, ошибки линковки, ошибки времени выполнения, логические ошибки.
Виды ошибок времени выполнения: внутренние и внешние ошибки. Методы борьбы с ошибками: макрос \texttt{assert}, использование глобальной переменной(\texttt{errno}), коды возврата и исключения. Преемущества и недостатки каждого из этих методов. Какие из этих методов желательно использовать для внутренних ошибок, а какие для внешних?


\item \textbf{assert}\\
Макрос \texttt{assert} и его применения для обнаружения ошибок.

\item \textbf{Коды возврата и класс \texttt{std::optional}}\\
Обработка ошибок с помощью кодов возврата. Примеры стандартных фуцнкий, использующих коды возврата.
Класс \texttt{optional} из стандартной библиотеки.
Методы класса \texttt{optional}:
\begin{itemize}
\item Конструкторы
\item Методы \texttt{value}, \texttt{has\_value}, \texttt{value\_or}.
\item Унарные операторы \texttt{*} и \texttt{->}
\item Оператор преобразования к значению типа \texttt{bool}.
\end{itemize}
Для чего можно применять \texttt{std::optional}? Использование класса \texttt{optional} для обработки ошибок с помощью кодов возврата.


\item \textbf{Исключения.}\\
Зачем нужны исключения, в чём их преимущество перед другими методами обработки ошибок?
Оператор \texttt{throw}, аргументы каких типов может принимать данный оператор. Что происходит после достижения программы оператора \texttt{throw}. Раскручивание стека. Блок \texttt{try-catch}. Что произойдёт, если выброшенное исключение не будет поймано? Стандартные классы исключений: \texttt{std::exception}, \texttt{std::runtime\_error}, \texttt{std::bad\_alloc}, \texttt{std::bad\_cast}, \texttt{std::logic\_error}. Почему желательно ловить стандартные исключение по ссылке на базовый класс \texttt{std::exception}? Использование \texttt{catch} для ловли всех типов исключений. Использование исключений в кострукторах, деструкторах, перегруженных операторах. Спецификатор \texttt{noexcept}. Гарантии безопасности исключений. Исключения при перемещении объектов. \texttt{move\_if\_noexcept}. Идиома \texttt{copy and swap}.

\end{enumerate}
\end{enumerate}







\begin{enumerate}

\item \textbf{Функциональные объекты}\\
Указатели на функции в алгоритмах STL. Функторы. Стандартные функторы: \texttt{std::less}, \texttt{std::greater}, \texttt{std::equal\_to}, \texttt{std::plus}, \texttt{std::minus}, \texttt{std::multiplies}. Основы лямбда-функций. Стандартные алгоритмы STL, принимающие функциональные объекты. Тип обёртка \texttt{std::function}. Шаблонная функция \texttt{std::bind}.

\item \textbf{Лямбда-функций}\\
Лямбда-функций. Объявление лямбда-функций. Передача их в другие функции. Преимущества лямбда-функций перед указателями на функции и функторами. Использование лямбда функций со стандартными алгоритмами \texttt{std::sort}, \texttt{std::transform}, \texttt{str::copy\_if}. Лямбда-захват. Захват по значению и по ссылке. Захват всех переменных области видимости по значению и по ссылке. Объявление новых переменных внутри захвата.







\item \textbf{Реализация вектора.}\\
Реализация своего вектора \texttt{mipt::Vector<T>} (аналога \texttt{std::vector<T>}). Нужно также предусмотреть итераторы этого вектора: \texttt{mipt::Vector<T>::iterator}, а также константные и обратные итераторы.\\
Методы такого вектора:
\begin{itemize}
\item Конструктор по умолчанию
\item Конструктор, принимающий количество элементов
\item Конструктор, принимающий количество элементов и значение элемента
\item Конструктор от \texttt{std::initializer\_list}.
\item Конструктор копирования
\item Конструктор перемещения
\item Деструктор
\item Оператор присваивания копирования
\item Оператор присваивания перемещения
\item Оператор взятия индекса (\texttt{operator[]})
\item Метод \texttt{at}, аналог метода \texttt{at} класса \texttt{std::vector}

\item Методы \texttt{size}, \texttt{capacity}, \texttt{empty}, \texttt{reserve}, \texttt{resize}, \texttt{shrink\_to\_fit}.
\item Методы \texttt{push\_back}, \texttt{emplace\_back}, \texttt{pop\_back}.

\item Методы для работы с итераторами \texttt{begin}, \texttt{end}, \texttt{rbegin}, \texttt{rend}.
\end{itemize}

Безопасность относительно исключений у такого вектора.




\item \textbf{Система типов языка \texttt{C++}.}\\
Система типов языка \texttt{C++}. Встроенные типы, массивы, структуры, объединения, перечисления, классы, указатели, ссылки, функциональные объекты (функции, указатели и ссылки на функции, функторы, лямбда-функции), указатели на члены класса, битовые поля. Вывод типа выражения с помощью \texttt{decltype}. Различие вывода с помощью \texttt{decltype}, \texttt{auto} и вывода шаблонных аргументов. Разложение типов (type decay) и когда он происходит.


\item \textbf{Приведение типов}\\
В чём недостатки приведения в стиле \texttt{C}? Оператор \texttt{static\_cast} и в каких случая он используется. Операторы \texttt{reinterpret\_cast} и \texttt{const\_cast} и в каких случая они используется. 




\item \textbf{Вычисления на этапе компиляции. \texttt{constexpr}} \\
Вычисление на этапе компиляции. В чём преимущества вычисления на этапе компиляции по сравнению с вычислением на этапе выполнения.  Ключевое слово \texttt{constexpr}. Что означает \texttt{constexpr} при объявлении переменной? Что означает \texttt{constexpr} при определении функции? Разница между \texttt{const} и \texttt{constexpr}. Ключевые слова \texttt{consteval} и \texttt{constinit}. \texttt{static\_assert}.


\item \textbf{Вычисления на этапе компиляции. Шаблонное метапрограммирование.} \\
Полная специализация шаблона. Частичная специализация шаблона. Что такое шаблонные метафункции и зачем они нужны? Использование специализации шаблона для написание следующих метафункций:
\begin{itemize}
\item \texttt{IsInt} - проверяет, является ли тип \texttt{T} типом \texttt{int}.
\item \texttt{IsIntegral} - проверяет, является ли тип \texttt{T} целочисленным типом.
\item \texttt{IsPointer} - проверяет, является ли тип \texttt{T} указателем.
\item \texttt{IsSame} - проверяет, являются ли 2 типа \texttt{T1} и \texttt{T2} одинаковыми.
\item \texttt{RemovePointer} - если тип \texttt{T} является указателем, то возвращает тип того, на что такой указатель указывает (то есть убирает одну "звёздочку" у типа).
\item \texttt{IsHasBegin} - проверяет, есть ли у типа \texttt{T} метод begin.
\end{itemize}
Что такое концепты, как их использовать и зачем они нужны?



\item \textbf{Универсальные ссылки}\\
Правила свёртки ссылок. Универсальные ссылки, чем они отличаются от lvalue и rvalue ссылок? Реализация функции \texttt{std::move}. Идеальная передача. Функция \texttt{std::forward}.

\end{enumerate}

\end{document}