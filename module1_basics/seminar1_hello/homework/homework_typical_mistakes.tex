\documentclass{article}
\usepackage[english,russian]{babel}
\usepackage{textcomp}
\usepackage{geometry}
  \geometry{left=2cm}
  \geometry{right=1.5cm}
  \geometry{top=1.5cm}
  \geometry{bottom=2cm}
\usepackage{tikz}
\usepackage{multicol}
\usepackage{hyperref}
\usepackage{amsmath}
\usepackage{listings}
\pagenumbering{gobble}

\lstset{
  language=C,
  basicstyle=\linespread{1.1}\ttfamily,
  columns=fixed,
  basewidth=0.5em,
  keywordstyle=\color{blue}\bfseries,
  commentstyle=\color{gray},
  stringstyle=\ttfamily\color{orange!50!black},
  showstringspaces=false,
  backgroundcolor=\color{white},
  breaklines=true,
  breakatwhitespace=true,
  xleftmargin=5mm,
  keepspaces = true,
  extendedchars=\true,
  tabsize=4,
  upquote=true,
}

\lstdefinestyle{csMiptBash}{
breaklines=true,
frame=tb,
language=bash,
breakatwhitespace=true,
alsoletter={*()"'0123456789.},
alsoother={\{\=\}},
basicstyle={\ttfamily},
keywordstyle={\bfseries},
literate={{=}{{{=}}}1},
prebreak={\textbackslash},
sensitive=true,
stepnumber=1,
tabsize=4,
morekeywords={echo, function},
otherkeywords={-, \{, \}},
literate={\$\{}{{{{\bfseries{}\$\{}}}}2,
upquote=true,
frame=none
}

\renewcommand{\thesubsection}{\arabic{subsection}}
\makeatletter
\def\@seccntformat#1{\@ifundefined{#1@cntformat}%
   {\csname the#1\endcsname\quad}
   {\csname #1@cntformat\endcsname}}
\newcommand\section@cntformat{}     
\newcommand\subsection@cntformat{Задача \thesubsection.\space} 
\newcommand\subsubsection@cntformat{\thesubsubsection.\space}
\makeatother

\begin{document}
\title{ДЗ. Типичные ошибки, возникающие при программировании на C. \vspace{-5ex}}\date{}\maketitle

\noindent В каждой задаче содержится пример кода на C, содержащий ошибку. Вам нужно определить в чём ошибка и приводит ли данная ошибка к неопределённому поведению.
Для того, чтобы сдать эту задачу нужно создать файл в формате \texttt{.txt} и, используя любой текстовый редактор, записать в него ответы в следующем формате (ответ ниже неверен):
\begin{verbatim}
1) Забыли поставить запятую в строке int main(). Ошибка приводит к неопределённому поведению.
2) ...
3) ...
\end{verbatim} 
После этого, файл нужно поместить в ваш репозиторий на github.


\subsection{}
Эта программа должна была просто печатать на экран \texttt{Hello World}. Но что-то идёт не так. Программа или не работает или печатает предупреждения (warnings).
\begin{lstlisting}
int main()
{
    printf("Hello World\n");
}
\end{lstlisting}

\subsection{}
Эта программа должна была считывать число и печатать его квадрат.
Но программа почему-то не работает.
\begin{lstlisting}
#include <stdio.h>
int main()
{
    int a = 0;
    scanf("%i", a);
    printf("%i\n", a * a);
}
\end{lstlisting}

\subsection{}
Данная простая программа должна была просто считывать одно число и проверять, равно ли это число 10.
\begin{itemize}
\item eсли число равно 10, то программа должна печатать Yes.
\item eсли число не равно 10, то программа должна печатать No.
\end{itemize}
Но почему-то программа всегда печатает \texttt{Yes}, независимо от числа.
\begin{lstlisting}
#include <stdio.h>
int main()
{
    int a;
    scanf("%i", &a);
    if (a = 10)
        printf("Yes\n");
    else
        printf("No\n");
}
\end{lstlisting}

\subsection{}
Эта программа должна была считывать число и печатать его квадрат, но программа почему-то работает не так как надо. Вместо того, чтобы просить 1 число программа почему-то просит 2 числа.
Но затем правильно печатает квадрат первого введённого числа.
\begin{lstlisting}
#include <stdio.h>
int main()
{
    int a = 0;
    scanf("%i\n", &a);
    printf("%i\n", a * a);
}
\end{lstlisting}


\subsection{}
Данная программа должна была просто считывать одно число и проверять, является ли это число положительным.  
\begin{itemize}
\item если число положительно, то программа должна печатать само число, а потом, в следующей строке печатать \texttt{Positive}.
\item если число отрицательно или равно нулю, то программа не должна ничего печатать.
\end{itemize}
Если число на входе положительно, то всё работает правильно.
Но если число отрицательное или равно нулю, то программа всё-равно почему-то печатает \texttt{Positive}.
\begin{lstlisting}
#include <stdio.h>
int main()
{
    int a;
    scanf("%i", &a);
    if (a > 0)
        printf("%i\n", a);
        printf("Positive\n");
}
\end{lstlisting}

\subsection{}
Данная программа должна была просто считывать одно число и проверять, является ли это число положительным.
\begin{itemize}
\item если число положительно, то программа должна печатать \texttt{Positive}.
\item если число отрицательно или равно нулю, то программа не должна ничего печатать.
\end{itemize}
Но почему-то программа всегда печатает \texttt{Positive}, независимо от числа.
\begin{lstlisting}
#include <stdio.h>
int main()
{
    int a;
    scanf("%i", &a);
    if (a > 0);
        printf("Positive\n");
}
\end{lstlisting}


\subsection{}
Данная программа должна была просто печатать на экран все целые числа от $1$ до $10$.
Но почему-то программа печатает число 1 бесконечное число раз.
Чтобы завершить работу программы можно использовать комбинацию клавиш  \texttt{Ctrl-C}.
\begin{lstlisting}
#include <stdio.h>
int main()
{
    int i = 1;
    while (i <= 10)
    {
        printf("%i ", i);
    }
    printf("\n");
}
\end{lstlisting}



\subsection{}
Данная программа должна была просто печатать на экран все целые числа от $1$ до $10$.
Но почему-то программа просто зависает и ничего не делает.
Чтобы завершить работу программы можно использовать комбинацию клавиш  \texttt{Ctrl-C}.
\begin{lstlisting}
#include <stdio.h>
int main()
{
    int i = 1;
    while (i <= 10);
    {
        printf("%i ", i);
        i += 1;
    }
    printf("\n");
}
\end{lstlisting}


\subsection{}
Данная программа должна была считывать число $n$ и высчитывать сумму чисел от $1$ до $n$ в цикле.
Эту сумму было бы проще и эффективней посчитать по формуле арифметической прогрессии, 
но мы посчитаем через цикл в образовательных целях.
В этой программе есть ошибка. Она может напечатать неправильный результат, а может и правильный.
Проверьте, правильный ли результат печатает эта программа на вашем компьютере.
\begin{lstlisting}
#include <stdio.h>
int main()
{
    int n;
    scanf("%i", &n);
    int i = 1;
    int sum;
    while (i <= n)
    {
        sum += i;
        i += 1;
    }
    printf("%i\n", sum);
}
\end{lstlisting}

\subsection{}
Данная программа также содержит ошибку.
\begin{lstlisting}
#include <stdio.h>
int main()
{
    int a[5] = {10, 20, 30, 40, 50};
    for (int i = 0; i <= 5; ++i)
    	printf("%i ", a[i]);
    printf("\n");
}
\end{lstlisting}


\end{document}