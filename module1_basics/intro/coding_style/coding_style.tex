\documentclass{article}
\usepackage[english,russian]{babel}
\usepackage{textcomp}
\usepackage{geometry}
  \geometry{left=2cm}
  \geometry{right=1.5cm}
  \geometry{top=1.5cm}
  \geometry{bottom=2cm}
\usepackage{tikz}
\usepackage{multicol}
\usepackage{hyperref}
\usepackage{amsmath}
\usepackage{listings}
\pagenumbering{gobble}

\lstset{
  language=C,
  basicstyle=\linespread{1.1}\ttfamily,
  columns=fixed,
  basewidth=0.5em,
  keywordstyle=\color{blue}\bfseries,
  commentstyle=\color{gray},
  stringstyle=\ttfamily\color{orange!50!black},
  showstringspaces=false,
  backgroundcolor=\color{white},
  breaklines=true,
  breakatwhitespace=true,
  xleftmargin=5mm,
  keepspaces = true,
  extendedchars=\true,
  tabsize=4,
  upquote=true,
}

\lstdefinestyle{csMiptBash}{
breaklines=true,
frame=tb,
language=bash,
breakatwhitespace=true,
alsoletter={*()"'0123456789.},
alsoother={\{\=\}},
basicstyle={\ttfamily},
keywordstyle={\bfseries},
literate={{=}{{{=}}}1},
prebreak={\textbackslash},
sensitive=true,
stepnumber=1,
tabsize=4,
morekeywords={echo, function},
otherkeywords={-, \{, \}},
literate={\$\{}{{{{\bfseries{}\$\{}}}}2,
upquote=true,
frame=none
}


\definecolor{correctcolor}{RGB}{247, 255, 247}
\definecolor{wrongcolor}{RGB}{255, 247, 247}


\begin{document}

\title{Правила оформления программ на языке C в данном курсе.}\date{}\maketitle
\subsection*{Стиль наименования идентификаторов}
Самые распространённые стили наименования идентификаторов это:
\begin{verbatim}
                 camelCase                 PascalCase              snake_case    
\end{verbatim}
Для языка C будем использовать \texttt{snake\_case} для почти всего, кроме названий новых типов. Для них будем использовать \texttt{Pascal\_Snake\_Case}.


\subsection*{Стиль расстановки скобок}
Можете использовать один из двух следующих вариантов расстановки скобок. Любые другие варианты будут считаться ошибкой оформления.
\vspace{-7mm}
\begin{multicols}{2}
\begin{lstlisting}[backgroundcolor = \color{correctcolor}]
int print_numbers(int n)
{
    while (n > 0)
    {
        cout << i << " ";
        n--;
    }
}
\end{lstlisting}

\begin{lstlisting}[backgroundcolor = \color{correctcolor}]
int print_numbers(int n) {
    while (n > 0) {
        cout << i << " ";
        n--;
    }
}
\end{lstlisting}

\end{multicols}



\subsection*{Стиль отступов для однострочной области видимости}
В случае если тело области видимости состоит из одной строки, то скобки можно не писать.
\vspace{-2mm}
\begin{multicols}{2}
\begin{lstlisting}[backgroundcolor = \color{correctcolor}]
int calculate_sum(int n)
{
    int sum = 0;
    for (int i = 0; i < n; ++i)
        sum += i;
        
    return sum;
}
\end{lstlisting}
\vfill
\begin{lstlisting}[backgroundcolor = \color{correctcolor}]
int calculate_sum(int n)
{
    int sum = 0;
    for (int i = 0; i < n; ++i)
    {
        sum += i;
    }
    return sum;
}
\end{lstlisting}
\end{multicols}
Исключение -- если одна из областей видимости оператора \texttt{if else} состоит из более чем одной строки, то все остальные области видимости этого операторы должны обрамляться скобочками.
\begin{multicols}{2}
\noindent
\begin{lstlisting}[backgroundcolor = \color{correctcolor}]
if (a > b)
{
    c = a;
}
else
{
    c = b;
}
\end{lstlisting}
\vfill
\begin{lstlisting}[backgroundcolor = \color{wrongcolor}]
if (a > b)
{
    c = a;
}
else
    c = b;
\end{lstlisting}
\end{multicols}

\subsection*{Пробелы после запятых}
После запятой или точки запятой обязательно ставится пробел. До запятой или точки с запятой пробел не ставится.

\vspace{-7mm}
\begin{multicols}{2}
\begin{lstlisting}[backgroundcolor = \color{correctcolor}]
int func(int a, int b)
{
    for (int i = a; i < b; ++i)
        printf("%i %i\n", a, b);
}
\end{lstlisting}

\begin{lstlisting}[backgroundcolor = \color{wrongcolor}]
int func(int a,int b)
{
    for (int i = a;i < b;++i)
        printf("%i %i\n",a,b);
}
\end{lstlisting}
\end{multicols}


\subsection*{Пробелы до и после бинарных операторов}
До и после бинарного оператора нужно обязательно ставить пробелы.
\vspace{-7mm}
\begin{multicols}{2}
\begin{lstlisting}[backgroundcolor = \color{correctcolor}]
int a = 10;
int b = 20;
int c = a + b;
int d;
d = a + b * c;
\end{lstlisting}

\begin{lstlisting}[backgroundcolor = \color{wrongcolor}]
int a = 10;
int b = 20;
int c = a+b;
int d;
d = a + b*c;
\end{lstlisting}
\end{multicols}


\subsection*{Отступы}
В качестве отступов используется ровно 4 пробела, а не знак табуляции.
Настройте ваш текстовый редактор при нажатии на \texttt{Tab} вставлял 4 пробела а не знак табуляции.
Причина, по которой желательно использовать пробелы, заключается в том, что символ табуляции может отображаться по-разному в различных программах и настройках редактора. Если вы случайно смешаете пробелы с табами в одном файле, то при открытии кода в другом редакторе все отступы "поедут" и структура кода нарушится.



\subsection*{Объявление несколько переменных на одной строке}
Запрещено объявлять несколько переменных на одной строке.
\vspace{-7mm}
\begin{multicols}{2}
\begin{lstlisting}[backgroundcolor = \color{correctcolor}]
int position = 100;
int velocity = 2;
\end{lstlisting}

\begin{lstlisting}[backgroundcolor = \color{wrongcolor}]
int position = 100, velocity = 2;
\end{lstlisting}
\end{multicols}

\subsection*{Объявление указателей}
При объявлении указателей звёздочка обязательно ставится рядом с названием типа.
\begin{multicols}{3}
\noindent
\begin{lstlisting}[backgroundcolor = \color{correctcolor}]
int a = 100;
int* p = &a;
\end{lstlisting}
\begin{lstlisting}[backgroundcolor = \color{wrongcolor}]
int a = 100;
int *p = &a;
\end{lstlisting}
\begin{lstlisting}[backgroundcolor = \color{wrongcolor}]
int a = 100;
int * p = &a;
\end{lstlisting}
\end{multicols}




\end{document}