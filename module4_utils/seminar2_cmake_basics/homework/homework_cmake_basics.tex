\documentclass{article}
\usepackage[utf8x]{inputenc}
\usepackage{ucs}
\usepackage{amsmath} 
\usepackage{amsfonts}
\usepackage{marvosym}
\usepackage{wasysym}
\usepackage{upgreek}
\usepackage[english,russian]{babel}
\usepackage{graphicx}
\usepackage{float}
\usepackage{textcomp}
\usepackage{hyperref}
\usepackage{geometry}
  \geometry{left=2cm}
  \geometry{right=1.5cm}
  \geometry{top=1cm}
  \geometry{bottom=2cm}
\usepackage{tikz}
\usepackage{ccaption}
\usepackage{multicol}
\usepackage{fancyvrb}
\usepackage{xcolor}

\hypersetup{
   colorlinks=true,
   citecolor=blue,
   linkcolor=black,
   urlcolor=blue
}

\usepackage{listings}
%\setlength{\columnsep}{1.5cm}
%\setlength{\columnseprule}{0.2pt}

\usepackage[absolute]{textpos}


\usepackage{colortbl,graphicx,tikz}
\definecolor{X}{rgb}{.5,.5,.5}

\renewcommand{\thesubsection}{\arabic{subsection}}

\begin{document}
\pagenumbering{gobble}
\lstset{
  language=C++,                % choose the language of the code
  basicstyle=\linespread{1.1}\ttfamily,
  columns=fixed,
  fontadjust=true,
  basewidth=0.5em,
  keywordstyle=\color{blue}\bfseries,
  commentstyle=\color{gray},
  stringstyle=\ttfamily\color{orange!50!black},
  showstringspaces=false,
  numbersep=5pt,
  numberstyle=\tiny\color{black},
  numberfirstline=true,
  stepnumber=1,                   % the step between two line-numbers.        
  numbersep=10pt,                  % how far the line-numbers are from the code
  backgroundcolor=\color{white},  % choose the background color. You must add \usepackage{color}
  showstringspaces=false,         % underline spaces within strings
  captionpos=b,                   % sets the caption-position to bottom
  breaklines=true,                % sets automatic line breaking
  breakatwhitespace=true,         % sets if automatic breaks should only happen at whitespace
  xleftmargin=.2in,
  extendedchars=\true,
  keepspaces = true,
}
\lstset{literate=%
   *{0}{{{\color{red!20!violet}0}}}1
    {1}{{{\color{red!20!violet}1}}}1
    {2}{{{\color{red!20!violet}2}}}1
    {3}{{{\color{red!20!violet}3}}}1
    {4}{{{\color{red!20!violet}4}}}1
    {5}{{{\color{red!20!violet}5}}}1
    {6}{{{\color{red!20!violet}6}}}1
    {7}{{{\color{red!20!violet}7}}}1
    {8}{{{\color{red!20!violet}8}}}1
    {9}{{{\color{red!20!violet}9}}}1
}
\newcommand\upquote[1]{\textquotesingle#1\textquotesingle}

\renewcommand{\thesubsection}{\arabic{subsection}}
\makeatletter
\def\@seccntformat#1{\@ifundefined{#1@cntformat}%
   {\csname the#1\endcsname\quad}%    default
   {\csname #1@cntformat\endcsname}}% enable individual control
\newcommand\section@cntformat{}     % section level 
\newcommand\subsection@cntformat{Задача \thesubsection.\space} % subsection level
\newcommand\subsubsection@cntformat{\thesubsubsection.\space} % subsubsection level
\makeatother


\makeatletter
\newcount\my@repeat@count
\newcommand{\myrepeat}[2]{%
  \begingroup
  \my@repeat@count=\z@
  \@whilenum\my@repeat@count<#1\do{#2\advance\my@repeat@count\@ne}%
  \endgroup
}
\makeatother

\title{Семинар: Основы CMake. Домашнее задание.\vspace{-5ex}}\date{}\maketitle
\subsection{Простой проект}
В папке \texttt{0simple\_project} содержится исходный код простой программы, которая создаёт изображение. Создайте проект CMake, который бы собирал этот проект.

\subsection{Библиотека}
В папке \texttt{1creating\_library} содержится код маленькой библиотеки для работы с изображениями.
Создайте проект CMake, который бы создавал статическую библиотеку.

\subsection{Использование библиотеки}
Перенесите файл библиотеки из прошлой задачи в папку \texttt{2linking\_library}. Создайте проект CMake, бы подключал статическую библиотеку \texttt{image} к файлу \texttt{main.cpp} и собирал исполняемый файл.

\subsection{Работа со списками}
В файле \texttt{3list/animals.cmake} содержится переменная \texttt{animals}, содержащая список животных. Напишите скрипты CMake, которые будут делать следующее:
\begin{itemize}
\item Печатать элементы списка на экран с помощью цикла. Каждое название животного в новой строке.
\item Печатать количество элементов списка
\item Печатать только те элементы списка, в который больше 4-х символов
\item Печатать только те элементы списка, которые начинаются на \textit{S}
\item Сортировать список и печатать его
\item Печатать элементы списка на экран в одну строку через пробел.
\item Создавать файл \texttt{data.txt} и сохранять в него список.
\item Создавать папку \texttt{animal\_files} и создавать в неё файлы с названиями элементов списка с расширением \texttt{.txt} (то есть \texttt{cat.txt}, \texttt{mouse.txt} и т. д.).
\end{itemize}

\subsection{Выбор файла исходного кода}
В папке \texttt{4choosing\_source/src} лежат три файла исходного кода. Каждый из файлов может компилироваться в исполняемый файл. Вам нужно дописать файл \texttt{4choosing\_source/CMakeLists.txt} так, чтобы можно было выбирать какой файл будет компилироваться во время конфигурации проекта в зависимости от значения переменной \texttt{ANIMAL\_TYPE}. То есть если мы перейдём в папку \texttt{build} и запустим:
\begin{verbatim}
cmake -DANIMAL_TYPE=Cat ..
cmake --build .
\end{verbatim}
то при сборке должен компилироваться файла \texttt{src/cat.cpp}, а получившийся исполняемый файл должен печатать на экран слово \texttt{Cat}.
Если же мы запустим:
\begin{verbatim}
cmake -DANIMAL_TYPE=Mouse ..
cmake --build .
\end{verbatim}
то при сборке должен компилироваться файла \texttt{src/mouse.cpp}, а получившийся исполняемый файл должен печатать на экран слово \texttt{Mouse}.
Если же мы не установим значение переменной \texttt{ANIMAL\_TYPE}, либо установим её значение не равное одному из значений: \texttt{Cat, Dog, Mouse}, то попытка конфигурации проекта должна приводить к фатальной ошибке.


\subsection{Функция печати переменной}
На языка CMake напишите функцию \texttt{print}, которая должна будет принимать на вход название переменной и печатать название переменной и её значение.
Например, следующий код:

\begin{Verbatim}[commandchars=\\\{\}]
\textcolor{teal}{set}(VAR \textcolor{olive}{"Cats and Dogs"})
print(VAR)
\end{Verbatim}
должен напечатать на экран 
\begin{verbatim}
VAR = "Cats and Dogs"
\end{verbatim}


\subsection{Передача в функцию}
На языке CMake написана функция \texttt{func}, которая печатает количество переданных ей аргументов и значения этих аргументов:
\begin{Verbatim}[commandchars=\\\{\}]
\textcolor{teal}{function}(func)
    \textcolor{teal}{message}(\textcolor{olive}{"Number of arguments =  $\{ARGC\}"})
    \textcolor{teal}{message}(\textcolor{olive}{"Arguments:"})
    \textcolor{teal}{foreach}(arg IN LISTS ARGV)
        \textcolor{teal}{message}($\{arg\})
    \textcolor{teal}{endforeach}()
    \textcolor{teal}{message}(\textcolor{olive}{"----------------------------"})
\textcolor{teal}{endfunction}()
\end{Verbatim}

Предположим, что есть переменная \texttt{X} определённая так:
\begin{Verbatim}[commandchars=\\\{\}]
\textcolor{teal}{set}(X \textcolor{olive}{"Tiger;Lion;Elephant"})
\end{Verbatim}
Что напечатает данная функция, если вызывать её следующим образом:

\begin{enumerate}
\item \texttt{func()}
\item \texttt{func(Cat)}
\item \texttt{func(Cat Mouse Dog)}
\item \texttt{func("Cat Mouse Dog")}
\item \texttt{func(Cat;Mouse;Dog)}
\item \texttt{func("Cat;Mouse;Dog")}
\item \texttt{func(X)}
\item \texttt{func(\$\{X\})}
\item \texttt{func("\$\{X\}")}
\item \texttt{func("Cat;Mouse;Dog" {} "\$\{X\}")}
\end{enumerate}

Для ответа на этот вопрос запишите результаты по каждому пункт в один текстовый файл.


\subsection{Сортировка пузырьком}
На языке CMake напишите функцию которая будет принимать список и возвращать отсортированную версию этого списка.





\end{document}
