\documentclass{article}
\usepackage[english,russian]{babel}
\usepackage{textcomp}
\usepackage{geometry}
  \geometry{left=2cm}
  \geometry{right=1.5cm}
  \geometry{top=1.5cm}
  \geometry{bottom=2cm}
\usepackage{tikz}
\usepackage{listings}
\usepackage{fancyvrb}
\usepackage{xcolor}
\pagenumbering{gobble}
\usepackage{multicol}

\lstdefinestyle{csMiptCppStyle}{
  language=C++,
  basicstyle=\linespread{1.1}\ttfamily,
  columns=fixed,
  fontadjust=true,
  basewidth=0.5em,
  keywordstyle=\color{blue}\bfseries,
  commentstyle=\color{gray},
  texcl=true,
  stringstyle=\ttfamily\color{orange!50!black},
  showstringspaces=false,
  numbersep=5pt,
  numberstyle=\tiny\color{black},
  numberfirstline=true,
  stepnumber=1,      
  numbersep=10pt,
  backgroundcolor=\color{white},
  showstringspaces=false,
  captionpos=b,
  breaklines=true
  breakatwhitespace=true,
  xleftmargin=.2in,
  extendedchars=\true,
  keepspaces = true,
  tabsize=4,
  upquote=true,
  frame=none,
  emph={if, else, elif, endif, include, define, defined},
  emphstyle={\color{blue}},
  literate=*{\#}{{\textcolor{blue}{\#}}}{1},
}


\lstdefinestyle{csMiptCppLinesStyle}{
  style=csMiptCppStyle,
  frame=lines,
}

\lstdefinestyle{csMiptCppBorderStyle}{
  style=csMiptCppStyle,
  framexleftmargin=5mm, 
  frame=shadowbox, 
  rulesepcolor=\color{gray}
}


\lstdefinestyle{csMiptBash}{
frame=none,
breaklines=true,
language=bash,
breakatwhitespace=true,
alsoletter={*()"'0123456789.},
alsoother={\{\=\}},
basicstyle={\ttfamily},
keywordstyle={\bfseries},
literate={{=}{{{=}}}1},
prebreak={\textbackslash},
sensitive=true,
stepnumber=1,
tabsize=4,
morekeywords={echo, function},
otherkeywords={-, \{, \}},
literate={\$\{}{{{{\bfseries{}\$\{}}}}2,
upquote=true,
}




\lstset{style=csMiptCppLinesStyle}
\lstset{literate={~}{{\raisebox{0.5ex}{\texttildelow}}}{1}}

\renewcommand{\thesubsection}{\arabic{subsection}}
\makeatletter
\def\@seccntformat#1{\@ifundefined{#1@cntformat}%
   {\csname the#1\endcsname\quad}%    default
   {\csname #1@cntformat\endcsname}}% enable individual control
\newcommand\section@cntformat{}     % section level 
\newcommand\subsection@cntformat{Задача \thesubsection.\space} % subsection level
\newcommand\subsubsection@cntformat{\thesubsubsection.\space} % subsubsection level
\makeatother

\begin{document}
\title{Семинар \#1: Компиляции. Домашнее задание. \vspace{-5ex}}\date{}\maketitle

В некоторых задачах этого домашнего задания необходимо будет создавать скрипты командной строки. Такие скрипты могут содержать несколько команд командной строки. При исполнении такого скрипта будут исполняться все эти команды. Если в задании говорится, что нужно создать скрипт, то подразумевается следующее:

\begin{itemize}
\item Если вы используете Windows:\\
Нужно будет создать batch-скрипт -- это файл с расширением \texttt{.bat}. Этот файл должен содержать все команды, которые обычно вводятся в командную строку. Запустить этот скрипт можно просто напечатав его имя в командной строке. Например, если файл называется \texttt{dog.bat}, то для его запуска нужно перейти в командной строке в папку, где этот файл находится, и написать следующее:
\begin{lstlisting}[style=csMiptBash]
dog.bat
\end{lstlisting}
\item Если вы используете Linux или macOS:\\
Нужно будет создать bash-скрипт -- это файл с расширением \texttt{.sh}, первая строка которого имеет вид:
\begin{lstlisting}[style=csMiptBash]
#!/bin/bash
\end{lstlisting}
После этого следуют все команды, которые обычно вводятся в командную строку.
Если файл называется \texttt{dog.sh}, то для его запуска нужно перейти в командной строке в папку, где этот файл находится, и написать следующее:
\begin{lstlisting}[style=csMiptBash]
chmod +x ./dog.sh        # изменяем права доступа, нужно выполнить только 1 раз 
./dog.sh                 # запускаем файл
\end{lstlisting}
\end{itemize} 

\noindent \textbf{Все материалы к этому заданию находятся в папке \texttt{seminar1\_libs/homework/code}}\\
\noindent \textbf{!! Делайте каждую задачу в отдельной папке}

\subsection{Раздельная компиляция: функции и структуры}
В папке \texttt{homework/code/01sep\_dynarray} лежит файл \texttt{dynarray.c} в котором находится простейшая реализация динамического массива, написанная на языке C. Вам нужно разделить этот файл на три части:
\begin{itemize}
\item Файл \texttt{dynarray.h} -- должен содержать все объявления, относящиеся к динамическому массиву.
\item Файл \texttt{dynarray.c} -- должен содержать определения всех функций, относящиихся к динамическому массиву.
\item Файл \texttt{main.c} -- должен содержать только функцию \texttt{main}.
\end{itemize}
Убедитесь, что программа состоящая из этих трёх файлов компилируется:
\begin{lstlisting}[style=csMiptBash]
gcc main.c dynarray.c
\end{lstlisting}


\subsection{Раздельная компиляция: класс}
В папке \texttt{02sep\_time} лежит файл \texttt{time.cpp} в котором находится класс \texttt{Time}. Вам нужно разделить этот файл на три части:
\begin{itemize}
\item Файл \texttt{time.hpp} -- должен содержать все объявления, относящиеся к классу \texttt{Time}.
\item Файл \texttt{time.cpp} -- должен содержать определения всех функций, относящихся к классу \texttt{Time}.
\item Файл \texttt{main.cpp} -- должен содержать только функцию \texttt{main}.
\end{itemize}
\subsection{Раздельная компиляция: шаблоны}
В папке \texttt{03sep\_template} лежит файл \texttt{tmath.cpp} в котором находится шаблонные функции \texttt{sum} и \texttt{mult}. Предполагается, что эти функции будут работать только с типами \texttt{int}, \texttt{unsigned int}, \texttt{float} и \texttt{double}. Вам нужно разделить этот файл на три части:
\begin{itemize}
\item Файл \texttt{tmath.hpp} -- должен содержать объявления шаблонных функции.
\item Файл \texttt{tmath.cpp} -- должен содержать определения шаблонных функций.
\item Файл \texttt{main.cpp} -- должен содержать только функцию \texttt{main}.
\end{itemize}


\subsection{Этапы компиляции}
В папке \texttt{04stages} лежит файл \texttt{hello.c}. Вам нужно написать скрипт, который бы создавал промежуточные файлы со всех этапов компиляции для этого файла. После запуска скрипта должны создаваться:
\begin{enumerate}
\item Файл \texttt{hello.i} -- код после этапа препроцессинга.
\item Файл \texttt{hello.s} -- код ассемблера после этапа компиляции.
\item Файл \texttt{hello.o} -- машинный код после этапа ассемблирования.
\item Исполняемый файл (\texttt{hello.exe} на Windows или просто \texttt{hello} на Linux)
\end{enumerate}


\subsection{Definitions}
В папке \texttt{05definitions} лежит программа, состоящая из пяти файлов. В файле \texttt{alice.c} содержится функция \texttt{cat}, которая печатает на экран либо строку \texttt{ALICE:CAT} либо строку \texttt{alice:cat} в зависимости от того, был ли определён макрос \texttt{UPPERCASE}. В файле \texttt{bob.c} содержится аналогичная функция \texttt{dog}. В файле \texttt{main.c} просто вызываются функции \texttt{cat} и \texttt{dog}. Вам нужно написать скрипт, который бы компилировал эту программу таким образом, чтобы на экран выводились строки:
\begin{lstlisting}[style=csMiptBash]
ALICE:CAT
bob:dog
\end{lstlisting}
Изменять сами файлы программы нельзя.


\subsection{Функции с одинаковым названием}
В папке \texttt{06function\_same\_name} лежит программа, состоящая из пяти файлов. В файле \texttt{alice.c} содержатся две функции: \texttt{cat} и \texttt{hello}, причём функция \texttt{cat} вызывает функцию \texttt{hello}. В файле \texttt{bob.c} содержатся две функции: \texttt{dog} и \texttt{hello}, причём функция \texttt{dog} вызывает функцию \texttt{hello}. В файле \texttt{main.c} просто вызываются функции \texttt{cat} и \texttt{dog}.
Но данная программа не компилируется из-за ошибки:
\begin{lstlisting}[style=csMiptBash]
multiple definition of `hello'
\end{lstlisting}
Исправьте данную программу, не меняя имена функций.

\subsection{Общая переменная}
В папке \texttt{07common\_variable} лежит программа, состоящая из пяти файлов. В файле \texttt{alice.c} содержится глобальная переменная \texttt{value} и функция \texttt{cat}, которая изменяет эту глобальную переменную. В файле \texttt{bob.c} также содержится глобальная переменная \texttt{value} и функция \texttt{dog}, которая изменяет эту глобальную переменную. В файле \texttt{main.c} просто вызываются функции \texttt{cat} и \texttt{dog} несколько раз и печатается значение переменной \texttt{value}.

Предполагалось, что переменная \texttt{value} будет одна на всю программу и функции \texttt{cat}, \texttt{dog} и \texttt{main} должны работать с одной и той же глобальной переменной. Однако, при попытке компиляции данной программы происходит ошибка:
\begin{lstlisting}[style=csMiptBash]
 multiple definition of `value'
\end{lstlisting}
Исправьте эту программу так, чтобы функции \texttt{cat}, \texttt{dog} и \texttt{main} работали с одной и той же глобальной переменной.


\subsection{Создание статической библиотеки}
В папке \texttt{08create\_static} лежит программа, состоящая из пяти файлов. Вам нужно будет написать скрипт, который бы создавал статическую библиотеку под названием \texttt{house}. Библиотека должна будет содержать скомпилированный код из файлов \texttt{alice.c} и \texttt{bob.c}. Название библиотеки должно иметь вид:
\begin{itemize}
\item \texttt{libhouse.a} -- если вы компилируете на Linux, macOS или если вы используете компилятор MinGW на Windows.
\item \texttt{house.lib} -- если вы компилируете на Windows, используя компилятор MSVC.
\end{itemize}

\subsection{Подключение статической библиотеки}
В этой задаче нужно будет подключить статическую библиотеку из предыдущей задачи. В папке\\ \texttt{09link\_static} лежит заготовка проекта. Вам нужно скопировать эту папку себе и добавьте файл статической библиотеки в подпапку \texttt{external/lib}. Структура проекта в папке \texttt{09link\_static} должна иметь вид:
\begin{verbatim}
09link_static
|--main.c
|--external
   |--include
      |--alice.h
      |--bob.h
   |--lib
      |--libhouse.a   (или другое имя, в зависимости от системы)
\end{verbatim}
В этой задаче вам нужно будет написать скрипт, который бы компилировал программу, подключая статическую библиотеку к исходному файлу \texttt{main.c}.

\subsection{Создание динамической библиотеки}
В папке \texttt{10create\_dynamic} лежит программа, состоящая из пяти файлов. Вам нужно будет написать скрипт, который бы создавал динамическую библиотеку под названием \texttt{house}. Библиотека должна будет содержать скомпилированный код из файлов \texttt{alice.c} и \texttt{bob.c}. Название библиотеки должно иметь вид:
\begin{itemize}
\item \texttt{libhouse.so} -- если вы компилируете на Linux.
\item \texttt{libhouse.dylib} -- если вы компилируете на macOS.
\item \texttt{house.dll} -- если вы компилируете на Windows (MSVC или MinGW).
\end{itemize}

\subsection{Подключение динамической библиотеки}
В этой задаче нужно будет подключить динамическую библиотеку из предыдущей задачи. В папке\\
\texttt{11link\_dynamic} лежит заготовка проекта. Вам нужно скопировать эту папку себе и добавьте файл динамической библиотеки в подпапку \texttt{external/lib}. Структура проекта в папке \texttt{11link\_dynamic} должна иметь вид:
\begin{verbatim}
11link_dynamic
|--main.c
|--external
   |--include
      |--alice.h
      |--bob.h
   |--lib
      |--libhouse.so   (или другое имя, в зависимости от системы)
\end{verbatim}
В этой задаче вам нужно будет написать 2 скрипта. Первый скрипт должен компилировать программу, подключая динамическую библиотеку к исходному файлу \texttt{main.c}. Второй скрипт должен запускать скомпилированный исполняемый файл.

\subsection{Подключение динамической библиотеки во время выполнения}
В предыдущей задаче динамическая библиотека подключалась во время запуска программы, но динамическую библиотеку можно подключить и в произвольный момент во время выполнения. Чтобы это сделать, нужно воспользоваться специальными библиотеками, представляемыми операционной системой.

\begin{itemize}
\item Если вы используете Linux или macOS, то вам нужно будет использовать библиотеку \texttt{dlfcn.h}.
\item Если вы используете Windows, то вам нужно будет использовать функции \texttt{LoadLibrary}, \texttt{GetProcAddress} и \texttt{FreeLibrary} из библиотеку \texttt{windows.h}.
\end{itemize}
Скопируйте папку \texttt{12link\_dynamic\_runtime} и добавьте в подпапку \texttt{external/lib} файл динамической библиотеке, созданный в задаче 10. После этого изменить файл \texttt{main.c} так, чтобы он подключал динамическую библиотеку во время выполнения.


\subsection{Общая функция}
В папке \texttt{13common\_function} лежит программа, написанная на C++, состоящая из шести файлов. В файле \texttt{hello.hpp} содержится функция \texttt{hello}. В файле \texttt{alice.cpp} содержатся функция \texttt{cat}, которая вызывает функцию \texttt{hello}. В файле \texttt{bob.cpp} содержатся функция \texttt{dog}, которая также вызывает функцию \texttt{hello}. В файле \texttt{main.cpp} просто вызываются функции \texttt{cat} и \texttt{dog}. Но данная программа не компилируется из-за ошибки:
\begin{lstlisting}[style=csMiptBash]
multiple definition of `hello'
\end{lstlisting}
Исправьте данную программу, не меняя имена функций и не используя \texttt{static}.



\subsection{Проект Image}
В папке \texttt{14image} содержится простой проект, написанный на C++ (стандарт C++20).
Проект содержит:
\begin{itemize}
\item Файл \texttt{main.cpp}, содержащий функцию \texttt{main}. Этот файл использует класс \texttt{Image}.
\item Файлы \texttt{image.cpp} и \texttt{image.hpp}, в которых содержится код класса \texttt{Image}.\\
Этот класс использует стороннюю библиотеку \texttt{stb}.
\item Сторонняя библиотека \texttt{stb} также находится в нашем проекте (в папке \texttt{external/stb}).
\end{itemize}
Результатом данного проекта должны являться:
\begin{itemize}
\item Статическая библиотека, содержащая код класса \texttt{Image}.
\item Исполняемый файл.
\end{itemize}
Напишите скрипт, который будет собирать проект (то есть создавать библиотеку и исполняемый файл). Менять код и структуру проекта нельзя.


\subsection{Подключение zlib}




\end{document}