\documentclass{article}
\usepackage[utf8x]{inputenc}
\usepackage{ucs}
\usepackage{amsmath} 
\usepackage{amsfonts}
\usepackage{upgreek}
\usepackage[english,russian]{babel}
\usepackage{graphicx}
\usepackage{float}
\usepackage{textcomp}
\usepackage{hyperref}
\usepackage{geometry}
  \geometry{left=2cm}
  \geometry{right=1.5cm}
  \geometry{top=1cm}
  \geometry{bottom=2cm}
\usepackage{tikz}
\usepackage{ccaption}
\usepackage{mathrsfs}
\usepackage{multicol}
\setlength{\multicolsep}{6pt}    % отступы сверху и снизу

\usepackage[shortlabels]{enumitem}
\usepackage{listings}
\lstset{
  language=C,                % choose the language of the code
  basicstyle=\linespread{1.1}\ttfamily,
  columns=fixed,
  fontadjust=true,
  basewidth=0.5em,
  keywordstyle=\color{blue}\bfseries,
  commentstyle=\color{gray},
  stringstyle=\ttfamily\color{orange!50!black},
  showstringspaces=false,
  %numbers=false,                   % where to put the line-numbers
  numbersep=5pt,
  numberstyle=\tiny\color{black},
  numberfirstline=true,
  stepnumber=1,                   % the step between two line-numbers.        
  numbersep=10pt,                  % how far the line-numbers are from the code
  backgroundcolor=\color{white},  % choose the background color. You must add \usepackage{color}
  showstringspaces=false,         % underline spaces within strings
  captionpos=b,                   % sets the caption-position to bottom
  breaklines=true,                % sets automatic line breaking
  breakatwhitespace=true,         % sets if automatic breaks should only happen at whitespace
  xleftmargin=.2in,
  extendedchars=\true,
  keepspaces = true,
}
\lstset{literate=%
   *{0}{{{\color{red!20!violet}0}}}1
    {1}{{{\color{red!20!violet}1}}}1
    {2}{{{\color{red!20!violet}2}}}1
    {3}{{{\color{red!20!violet}3}}}1
    {4}{{{\color{red!20!violet}4}}}1
    {5}{{{\color{red!20!violet}5}}}1
    {6}{{{\color{red!20!violet}6}}}1
    {7}{{{\color{red!20!violet}7}}}1
    {8}{{{\color{red!20!violet}8}}}1
    {9}{{{\color{red!20!violet}9}}}1
}


\begin{document}
\pagenumbering{gobble}

\section*{Модуль 4. Вопросы.}
\begin{enumerate}

\item \textbf{Компиляция и линковка}
\begin{enumerate}[a.]

\item \textbf{Этапы компиляции}\\
Что такое файл исходного кода и исполняемый файл? Этапы компиляции: препроцессинг, компиляция, ассемблирование, линковка. Что происходит на каждом из этапе компиляции? Как выполнить только часть этапов компиляции?

\item \textbf{Раздельная компиляция}\\
В чем преимущества разделения программы на несколько файлов? В чём преимущества и недостатки раздельной компиляции? Как разделить программу на файлы так, чтобы происходила раздельная компиляция? Как программа компилируется при раздельной компиляции?
 
\item \textbf{Заголовочные файлы}\\
Что такое заголовочные файлы (header-файлы)? Что обычно хранится в заголовочных файлах? Что делает директива препроцессора \texttt{\#include}? Проблема множественного включения. Стражи включения и директива \texttt{\#pragma once}.

\item \textbf{Имена}\\
Что такое единица трансляции?  Понятия объявления (англ. \textit{declaration}) и определения (англ. \textit{definition}) сущностей. Объявление функций. Определение функций. Объявление глобальный переменных. Определение глобальных переменных. Объявление и определение структур. Понятие \textit{имя} в контексте линковки на языке C. Понятие \textit{символ} в контексте линковки на языке C. Внутреннее связывание, внешнее связывания и отсутствие связывания. Ключевые слова \texttt{static} и \texttt{extern}. Ключевое слово \texttt{inline}. Использования утилит \texttt{nm} и \texttt{objdump} для просмотра имён в скомпилированных файлах.

\item \textbf{Опции компиляции}
\begin{itemize}
\item Опция для задания названия выходного файла.
\item Опции для выполнения только первых этапов компиляции
\item Опции для выбора стандарта языка
\item Опции для работы с предупреждениями (\texttt{-Wall}, \texttt{-Wextra}, \texttt{-Werror})
\item Опции для оптимизации кода
\item Опция для включения информации для дебаггера
\item Опция для определения макросов (определений компиляции)
\item Опция \texttt{-DNDEBUG}
\item Опции для подключения библиотек
\end{itemize}

\item \textbf{Библиотеки}\\
Что такое библиотека? Виды библиотек: header-only библиотеки, статические библиотеки, динамические библиотеки. В чём различия между этими видами библиотек? В чём преимущества и недостатки каждого из видов библиотек?

\item \textbf{Статические библиотеки}\\
Как создать статическую библиотеку? Утилита \texttt{ar}. Как подключить статическую библиотеку, с помощью компилятора \texttt{gcc}? Опции компилятора \texttt{-I}, \texttt{-L} и \texttt{-l}. Характерные расширения файлов статических библиотек на Linux (gcc) и Windows (MinGW и MSVC). 

\item \textbf{Динамические библиотеки}\\
В чём главная разница между статическими и динамическими библиотеками? Как создать динамическую библиотеку? Что делает опция \texttt{-fPIC}? Характерные расширения файлов динамических библиотек на Linux (gcc) и Windows (MinGW и MSVC). Подключение динамической библиотеки при запуске программы. Опции компилятора \texttt{-I}, \texttt{-L} и \texttt{-l}. Подключение динамической библиотеки во время выполнения программы. Библиотека \texttt{dlfcn.h}. Функции \texttt{dlopen}, \texttt{dlsym} и \texttt{dlclose}.


\item \textbf{Особенности компиляции и линковки на языке С++ по сравнению с языком C}
\begin{itemize}
\item Раздельная компиляция при наличии в коде классов и шаблонов.
\item Манглирования имён. \texttt{extern "C"}.
\item ODR. Использования ключевого слова \texttt{inline} для обхода ODR.
\item Инициализация статических полей класса.
\end{itemize}

\end{enumerate}


\newpage
\item \textbf{Основы CMake}
\begin{enumerate}[a.]
\item \textbf{Системы сборки}\\
Задачи, которые решают системы сборки. Примеры систем сборки. Генераторы систем сборки.
Преимущества генераторов систем сборки по сравнению с обычными системами сборки. Примеры генераторов систем сборки. Как задать генератор в CMake? Опция \texttt{-G} программы \texttt{cmake}.

\item \textbf{Простейшие проекты на CMake}\\
Файл \texttt{CMakeLists.txt}. Команды:
\begin{itemize}
\item \texttt{cmake\_minimum\_required}
\item \texttt{project}
\item \texttt{add\_executable}
\end{itemize}
Создание и подключение библиотек в CMake. Команды:
\begin{itemize}
\item \texttt{add\_library}
\item \texttt{target\_link\_libraries}
\end{itemize}


\item \textbf{Таргеты}\\
Понятие \textit{таргет} в CMake. 
Подключение файлов к таргетам с помощью команды:
\begin{itemize}
\item \texttt{target\_sources}
\end{itemize}

Подключение опций к таргетам с помощью команд:
\begin{itemize}
\item \texttt{target\_include\_directories}
\item \texttt{target\_link\_directories}
\item \texttt{target\_link\_libraries} (не путать с другой командой с таким же названием)
\item \texttt{target\_compile\_definitions}
\item \texttt{target\_compile\_features}
\item \texttt{target\_compile\_options}
\item \texttt{target\_link\_options}
\end{itemize}

Использование ключевых слов \texttt{PUBLIC}, \texttt{PRIVATE} и \texttt{INTERFACE} при подключении опций к таргетам и при подключении таргетов друг к другу.

\item \textbf{Переменные CMake}\\
Язык CMake. Создание переменных в CMake. Команда \texttt{set}. Какого типа может быть переменная в CMake? Использование переменных. Переменные-строки. Команда \texttt{string} и её опции:
\noindent \begin{multicols}{3}
\begin{itemize}
\item \texttt{FIND}
\item \texttt{REPLACE}
\item \texttt{APPEND}
\item \texttt{JOIN}
\item \texttt{TOLOWER}
\item \texttt{TOUPPER}
\item \texttt{LENGTH}
\item \texttt{SUBSTRING}
\item \texttt{COMPARE}
\end{itemize}
\end{multicols}
\noindent Переменные-списки. Команда \texttt{list} и её опции:
\noindent \begin{multicols}{3}
\begin{itemize}
\item \texttt{LENGTH}
\item \texttt{GET}
\item \texttt{JOIN}
\item \texttt{FIND}
\item \texttt{APPEND}
\item \texttt{INSERT}
\item \texttt{REMOVE\_ITEM}
\item \texttt{REMOVE\_AT}
\item \texttt{SORT}
\end{itemize}
\end{multicols}

\item \textbf{Команда \texttt{if} и циклы}\\
Как работает команда \texttt{if}. Какие строки считаются истинными, а какие -- ложными?
Цикл \texttt{while}. Цикл \texttt{foreach}. Разновидности цикла \texttt{foreach}: \texttt{RANGE}, \texttt{IN ITEMS} и \texttt{IN LISTS}.

\item \textbf{Функции CMake}\\
Сколько аргументов могут принимать функции в CMake? Команда \texttt{function} для создания функций. Передача аргументов в функции. Переменные \texttt{ARGC}, \texttt{ARGV} и \texttt{ARGN}. Возврат из функций. \texttt{PARENT\_SCOPE}.

\item \textbf{Работа с файлами в CMake. Команда \texttt{file} и её опции.}
\begin{multicols}{3}
\begin{itemize}
\item \texttt{READ}
\item \texttt{STRINGS}
\item \texttt{WRITE}
\item \texttt{APPEND}
\item \texttt{GLOB}
\item \texttt{MAKE\_DIRECTORY}
\item \texttt{COPY\_FILE}
\item \texttt{SIZE}
\item \texttt{DOWNLOAD}

\end{itemize}
\end{multicols}

\end{enumerate}





\newpage
\item \textbf{CMake: структура проекта, кэшированные переменные, конфиги, генераторные выражения}
\begin{enumerate}[a.]
\item \textbf{Разделение кода CMake на несколько директорий}\\
Поддиректории в CMake. Команда \texttt{add\_subdirectory}. 
Область видимости переменных, объявленных в поддиректории и вне поддиректории. Опция \texttt{PARENT\_SCOPE} команды \texttt{set}. Переменные:
\begin{itemize}
\item \texttt{CMAKE\_SOURCE\_DIR}.
\item \texttt{CMAKE\_BINARY\_DIR}.
\item \texttt{CMAKE\_CURRENT\_SOURCE\_DIR}
\item \texttt{CMAKE\_CURRENT\_BINARY\_DIR}
\end{itemize}

\item \textbf{Модули}\\
Модули CMake. Скриптовый режим. \texttt{cmake -P}. Подключение модуля. Команда \texttt{include}. Переменные:
\begin{itemize}
\item \texttt{CMAKE\_MODULE\_PATH}.
\item \texttt{CMAKE\_CURRENT\_LIST\_DIR}
\item \texttt{CMAKE\_CURRENT\_LIST\_FILE}
\end{itemize}
Область видимости переменных, объявленных в модуле и вне модуля.


\item \textbf{Кэшированные переменные}\\
Что такое кэшированная переменная CMake и чем она отличается от обычной? Зачем нужны кэшированные переменные? Создание кэшированных переменных с помощью команды \texttt{set}. Создание кэшированных переменных с помощью команды \texttt{option}. Как изменить уже созданную кэшированную переменную? Опция \texttt{FORCE} команды \texttt{set}. Файл \texttt{CMakeCache.txt}. Опция \texttt{-D} программы \texttt{cmake}. Программы \texttt{ccmake} и \texttt{cmake-gui}. "Тип"{} кэшированной переменной и как он используется? Совпаденение имени обычной и кэшированной переменной.


\item \textbf{Предопределённые переменные}\\
Переменные, хранящие информацию о проекте:
\begin{itemize}
\item \texttt{PROJECT\_NAME}
\item \texttt{PROJECT\_SOURCE\_DIR}
\item \texttt{PROJECT\_BINARY\_DIR}
\item \texttt{PROJECT\_IS\_TOP\_LEVEL}
\end{itemize}

Переменные, задающие глобальные опции компиляции:
\begin{itemize}
\item \texttt{CMAKE\_C\_FLAGS}
\item \texttt{CMAKE\_CXX\_FLAGS}
\item \texttt{CMAKE\_EXE\_LINKER\_FLAGS}
\end{itemize}

Переменные, предоставляющие информацию о платформе:
\begin{multicols}{2}
\begin{itemize}
\item \texttt{CMAKE\_C\_COMPILER}
\item \texttt{CMAKE\_CXX\_COMPILER}
\item \texttt{CMAKE\_GENERATOR}
\item \texttt{CMAKE\_SYSTEM\_NAME}
\item \texttt{UNIX}
\item \texttt{APPLE}
\item \texttt{WIN32}
\item \texttt{MSVC}
\end{itemize}
\end{multicols}
Информация о самой программе \texttt{cmake}:
\begin{itemize}
\item \texttt{CMAKE\_ROOT}
\item \texttt{CMAKE\_VERSION}
\end{itemize}

\item \textbf{Свойства}\\
Что такое свойства в CMake? Чем свойства отличаются от переменных? Команды \texttt{get\_property} и \texttt{set\_property}. Глобальные свойства. Свойства директорий:
\begin{multicols}{2}
\begin{itemize}
\item \texttt{VARIABLES}
\item \texttt{CACHE\_VARIABLES}
\item \texttt{SUBDIRECTORIES}
\item \texttt{BUILDSYSTEM\_TARGETS}
\item \texttt{IMPORTED\_TARGETS}
\item \texttt{TESTS}
\item \texttt{PARENT\_DIRECTORY}
\end{itemize}
\end{multicols}

\newpage
Свойства таргетов:
\begin{multicols}{2}
\begin{itemize}
\item \texttt{INCLUDE\_DIRECTORIES}
\item \texttt{COMPILE\_DEFINITIONS}
\item \texttt{COMPILE\_FEATURES}
\item \texttt{COMPILE\_OPTIONS}
\item \texttt{LINK\_DIRECTORIES}
\item \texttt{LINK\_LIBRARIES}
\item \texttt{LINK\_OPTIONS}
\item \texttt{INTERFACE\_INCLUDE\_DIRECTORIES}
\item \texttt{INTERFACE\_COMPILE\_DEFINITIONS}
\item \texttt{INTERFACE\_COMPILE\_FEATURES}
\item \texttt{INTERFACE\_COMPILE\_OPTIONS}
\item \texttt{INTERFACE\_LINK\_DIRECTORIES}
\item \texttt{INTERFACE\_LINK\_LIBRARIES}
\item \texttt{INTERFACE\_LINK\_OPTIONS}
\item \texttt{SOURCES}
\item \texttt{TYPE}
\item \texttt{IMPORTED}
\item \texttt{OUTPUT\_NAME}
\item \texttt{OUTPUT\_NAME\_<CONFIG>}
\item \texttt{ARCHIVE\_OUTPUT\_DIRECTORY}
\item \texttt{ARCHIVE\_OUTPUT\_DIRECTORY\_<CONFIG>}
\item \texttt{LIBRARY\_OUTPUT\_DIRECTORY}
\item \texttt{LIBRARY\_OUTPUT\_DIRECTORY\_<CONFIG>}
\item \texttt{RUNTIME\_OUTPUT\_DIRECTORY}
\item \texttt{RUNTIME\_OUTPUT\_DIRECTORY\_<CONFIG>}
\item \texttt{C\_STANDARD}
\item \texttt{CXX\_STANDARD}
\end{itemize}
\end{multicols}
Свойства файлов исходного кода. Свойства тестов.

\item \textbf{Типы сборки}\\
Основные типы сборки:
\begin{itemize}
\item (Пустой тип сборки)
\item \texttt{Release}
\item \texttt{Debug}
\item \texttt{RelWithDebInfo}
\item \texttt{MinSizeRel}
\end{itemize}
Как тип сборки влияет на опции компиляции? 
Одноконфигурационные и мультиконфигурационные генераторы. Как задать тип сборки в случае одноконфигурационного генератора и в случае мультиконфигурационного генератора? Переменные:
\begin{itemize}
\item \texttt{CMAKE\_BUILD\_TYPE}
\item \texttt{CMAKE\_CONFIGURATION\_TYPES}
\end{itemize}
Глобальное свойство:
\begin{itemize}
\item \texttt{GENERATOR\_IS\_MULTI\_CONFIG}
\end{itemize}
Как написать код, который бы работал для одноконфигурационных и мультиконфигурационных генераторов.
Создание пользовательского типа сборки. Кэшированные переменные:
\begin{itemize}
\item \texttt{CMAKE\_CXX\_FLAGS\_<CONFIG>}
\item \texttt{CMAKE\_C\_FLAGS\_<CONFIG>}
\item \texttt{CMAKE\_EXE\_LINKER\_FLAGS\_<CONFIG>}
\end{itemize}
\item \textbf{Генераторные выражения}\\
Этапы сборки CMake-проекта:
\begin{enumerate}
\item Этап конфигурации
\item Этап генерации
\item Этап сборки
\end{enumerate}

Генераторные выражения. Привести пример применения генераторного выражения. Какие команды CMake поддерживают генераторные выражения? Основные виды генераторных выражений:
\begin{itemize}
\item \texttt{\$<условие:значение>}
\item \texttt{\$<BOOL:значение>}
\item Логические операции \texttt{OR}, \texttt{AND},  \texttt{NOT}.
\item \texttt{\$<CONFIG:тип\_сборки>}
\item \texttt{\$<PLATFORM\_ID:имя\_платформы>}
\item \texttt{\$<TARGET\_FILE:таргет>}
\end{itemize}


\end{enumerate}





\newpage
\item \textbf{CMake: подключение сторонних библиотек}
\begin{enumerate}[a.]
\item \textbf{Ручное подключение сторонней библиотеки}\\
Как подключить библиотеку, если есть только файлы самой библиотеки без скриптов сборки?
Преимущества и недостатки этого метода подключения сторонних библиотек.

\item \textbf{Подключение с помощью \texttt{add\_subdirectory}}\\
Как подключить другой CMake-проект с помощью \texttt{add\_subdirectory}?
Преимущества и недостатки этого метода подключения сторонних библиотек.

\item \textbf{\texttt{find\_package}}\\
Команда \texttt{find\_file}. Алгоритм поиска файла в системе при использовании \texttt{find\_file}. Переменные:
\begin{itemize}
\item \texttt{<ИмяПакета>\_ROOT}
\item \texttt{CMAKE\_PREFIX\_PATH}
\item \texttt{CMAKE\_INCLUDE\_PATH}
\item \texttt{CMAKE\_FRAMEWORK\_PATH}
\item \texttt{CMAKE\_SYSTEM\_PREFIX\_PATH}
\end{itemize}
Опции \texttt{HINTS} и \texttt{PATHS}. Команды \texttt{find\_library} и \texttt{find\_program}.\\
Команда \texttt{find\_package} и её опции:
\begin{itemize}
\item \texttt{REQUIRED}
\item \texttt{COMPONENTS}
\end{itemize}
Два режима работы команды \texttt{find\_package}:
\begin{enumerate}
\item Режим поиска модуля:\\
Какой файл ищет CMake в этом режиме? Алгоритм поиска файла. Переменная \texttt{CMAKE\_MODULE\_PATH}.
\item Режим поиска конфигурационного файла:\\
Какой файл ищет CMake в этом режиме? Алгоритм поиска файла. Переменные:
\begin{itemize}
\item \texttt{<ИмяПакета>\_ROOT}
\item \texttt{CMAKE\_PREFIX\_PATH}
\item \texttt{CMAKE\_FRAMEWORK\_PATH}
\item \texttt{CMAKE\_APPBUNDLE\_PATH}
\end{itemize}
\end{enumerate}

Переменные, создаваемые командой \texttt{find\_package}:
\begin{itemize}
\item \texttt{<ИмяПакета>\_FOUND}
\item \texttt{<ИмяПакета>\_LIBRARIES}
\end{itemize}
Импортированные таргеты, создаваемые командой \texttt{find\_package}.\\
Преимущества и недостатки \texttt{find\_package} для подключения сторонних библиотек.


\item \textbf{\texttt{ExternalProject}}\\
Команда \texttt{ExternalProject\_Add} и её опции:
\begin{itemize}
\item \texttt{GIT\_REPOSITORY} 
\item \texttt{GIT\_TAG}
\item \texttt{DOWNLOAD\_COMMAND}
\item \texttt{BUILD\_COMMAND}
\item \texttt{INSTALL\_COMMAND}
\end{itemize}
Преимущества и недостатки этого метода подключения сторонних библиотек.

\item \textbf{\texttt{FetchContent}}\\
Команда \texttt{FetchContent\_Declare} и её опции:
\begin{itemize}
\item \texttt{GIT\_REPOSITORY} 
\item \texttt{GIT\_TAG}
\end{itemize}
Команда \texttt{FetchContent\_MakeAvailable}.
Чем \texttt{FetchContent} отличается от \texttt{ExternalProject}?
Преимущества и недостатки этого метода подключения сторонних библиотек.

\item \textbf{Пакетный менеджер \texttt{conan}}\\
Пакетные менеджеры для библиотеки на языках C и C++. Использование пакетного менеджера \texttt{conan} совместно с \texttt{find\_package}. Преимущества и недостатки этого метода подключения сторонних библиотек.

\end{enumerate}




\item \textbf{Тестирование}
\begin{enumerate}[a.]
\item \textbf{Основы тестирования}\\
Что такое тесты? Написание тестов без использования сторонних библиотек. Юнит-тестирование.


\item \textbf{CTest}\\
Команда \texttt{enable\_testing}. Команда \texttt{add\_test} и её опции \texttt{NAME}, \texttt{COMMAND} и \texttt{WORKING\_DIRECTORY}. Свойства тестов:
\begin{itemize}
\item \texttt{TIMEOUT}
\item \texttt{WILL\_FAIL}
\item \texttt{FAIL\_REGULAR\_EXPRESSION}
\item \texttt{PASS\_REGULAR\_EXPRESSION}
\item \texttt{LABELS}
\item \texttt{COST}
\end{itemize}

Программа \texttt{ctest}. Опции команды \texttt{ctest}: 
\begin{itemize}
\item Опция \texttt{-N}
\item Опции \texttt{-R} и \texttt{-E}
\item Опция \texttt{-j}
\item Опция \texttt{--repeat-until-fail}
\item Опция \texttt{--timeout}
\end{itemize}

\item \textbf{Подключение GoogleTest}\\
Подключение библиотеки \texttt{Google Test} с помощью CMake.
Таргеты \texttt{gtest} и \texttt{gtest\_main}.

\item \textbf{GoogleTest}\\
Создание простых тестов. Макрос \texttt{TEST}. Наборы тестов.
\texttt{EXPECT\_} и \texttt{ASSERT\_} проверки. В чём отличие между ними?
Проверки:
\begin{itemize}
\item \texttt{EXPECT\_TRUE}, \texttt{EXPECT\_FALSE}
\item \texttt{EXPECT\_EQ}, \texttt{EXPECT\_LE} и т. д.
\item \texttt{EXPECT\_STREQ}
\item \texttt{EXPECT\_FLOAT\_EQ}, \texttt{EXPECT\_NEAR}
\item \texttt{EXPECT\_THROW}, \texttt{EXPECT\_ANY\_THROW}, \texttt{EXPECT\_NO\_THROW}
\item \texttt{EXPECT\_DEATH}, \texttt{EXPECT\_EXIT}
\end{itemize}
Фикстуры (англ. \textit{fixtures}). Макрос \texttt{TEST\_F}. Отключение тестов с помощью \texttt{DISABLED\_}.

\item \textbf{GoogleMock}\\
Мок-объекты. Создание мок-классов. Макрос \texttt{MOCK\_METHOD}. Задание ожиданий. Макрос \texttt{EXPECT\_CALL}. Методы \texttt{Times}, \texttt{WillOnce} и \texttt{WillRepeatedly}.
Кардинальности:
\begin{itemize}
\item \texttt{testing::AnyNumber}
\item \texttt{testing::AtLeast}
\item \texttt{testing::AtMost}
\item \texttt{testing::Between}
\item \texttt{testing::Exactly}
\end{itemize}

Матчеры:
\begin{multicols}{2}
\begin{itemize}
\item \texttt{testing::\_}
\item \texttt{testing::Any<T>}
\item \texttt{testing::Eq}, \texttt{testing::Gt}, и т. д.
\item \texttt{testing::Contains}
\item \texttt{testing::Truly}
\item \texttt{testing::AllOf}
\item \texttt{testing::AnyOf}
\item \texttt{testing::ElementsAre}
\end{itemize}
\end{multicols}

Действия:
\begin{multicols}{2}
\begin{itemize}
\item \texttt{testing::Return}
\item \texttt{testing::ReturnNew}
\item \texttt{testing::SetArgReferee}
\item \texttt{testing::Throw}
\item \texttt{testing::Invoke}
\item \texttt{testing::DoAll}
\end{itemize}
\end{multicols}

\end{enumerate}




\end{enumerate}


\end{document}