\documentclass{article}
\usepackage[english,russian]{babel}
\usepackage{textcomp}
\usepackage{geometry}
  \geometry{left=2cm}
  \geometry{right=1.5cm}
  \geometry{top=1.5cm}
  \geometry{bottom=2cm}
\usepackage{tikz}
\usepackage{multicol}
\usepackage{hyperref}
\usepackage{listings}
\pagenumbering{gobble}

\lstdefinestyle{csMiptCppStyle}{
  language=C++,
  basicstyle=\linespread{1.1}\ttfamily,
  columns=fixed,
  fontadjust=true,
  basewidth=0.5em,
  keywordstyle=\color{blue}\bfseries,
  commentstyle=\color{gray},
  texcl=true,
  stringstyle=\ttfamily\color{orange!50!black},
  showstringspaces=false,
  numbersep=5pt,
  numberstyle=\tiny\color{black},
  numberfirstline=true,
  stepnumber=1,      
  numbersep=10pt,
  backgroundcolor=\color{white},
  showstringspaces=false,
  captionpos=b,
  breaklines=true
  breakatwhitespace=true,
  xleftmargin=.2in,
  extendedchars=\true,
  keepspaces = true,
  tabsize=4,
  upquote=true,
}


\lstdefinestyle{csMiptCppLinesStyle}{
  style=csMiptCppStyle,
  frame=lines,
}

\lstdefinestyle{csMiptCppBorderStyle}{
  style=csMiptCppStyle,
  framexleftmargin=5mm, 
  frame=shadowbox, 
  rulesepcolor=\color{gray}
}


\lstdefinestyle{csMiptBash}{
breaklines=true,
frame=tb,
language=bash,
breakatwhitespace=true,
alsoletter={*()"'0123456789.},
alsoother={\{\=\}},
basicstyle={\ttfamily},
keywordstyle={\bfseries},
literate={{=}{{{=}}}1},
prebreak={\textbackslash},
sensitive=true,
stepnumber=1,
tabsize=4,
morekeywords={echo, function},
otherkeywords={-, \{, \}},
literate={\$\{}{{{{\bfseries{}\$\{}}}}2,
upquote=true,
frame=none
}




\lstset{style=csMiptCppLinesStyle}
\lstset{literate={~}{{\raisebox{0.5ex}{\texttildelow}}}{1}}


\renewcommand{\thesection}{\arabic{section}}
\makeatletter
\def\@seccntformat#1{\@ifundefined{#1@cntformat}%
   {\csname the#1\endcsname\quad}%    default
   {\csname #1@cntformat\endcsname}}% enable individual control
\newcommand\section@cntformat{Часть \thesection:\space}
\makeatother




\begin{document}
\title{Семинар \#4: CMake: Подключение библиотек. \vspace{-5ex}}
\date{}\maketitle

\section{Наивное подключение зависимостей}
\subsection*{Подключение вручную}
\subsection*{Подключение с помощью \texttt{add\_subdirectory}}
\subsection*{Подключение с помощью \texttt{git submodules}}


\section{Поиски файлов}
\subsection*{\texttt{find\_file}}
\texttt{CMAKE\_PREFIX\_PATH}.
\subsection*{\texttt{find\_program} и \texttt{find\_library}}


\section{Поиски пакетов, \texttt{find\_package}}
Как просмотреть добавленные таргеты?\\
\texttt{<PACKAGE>\_LIBRARIES}\\
\texttt{<PACKAGE>\_TARGETS}\\
Просмотр \texttt{IMPORTED\_TARGETS}.\\

\subsection*{Module mode}
Переменная \texttt{CMAKE\_MODULE\_PATH}.


\subsection*{Config mode}
Алгоритм поиска скрипта в Config mode.\\
\texttt{CMAKE\_PREFIX\_PATH}.

\subsection*{Опции команды find\_package}
\texttt{EXACT}\\
\texttt{QUIET}\\ 
\texttt{MODULE}\\
\texttt{REQUIRED}\\
\texttt{COMPONENTS}
             
\section{ExternalProject}
\texttt{ExternalProject\_Add}. Супербилд.\\
\texttt{GIT\_REPOSITORY}, \texttt{GIT\_TAG}.

\section{FetchContent}
\texttt{FetchContent\_Declare}.\\
\texttt{FetchContent\_MakeAvailable}.\\
\texttt{GIT\_REPOSITORY}, \texttt{GIT\_TAG}.

\section{Toolchain-файлы}
\texttt{CMAKE\_TOOLCHAIN\_FILE}\\

\noindent
\texttt{CMAKE\_SYSTEM\_NAME} --	Целевая ОС (Linux, Windows, Android и т.д.).\\
\texttt{CMAKE\_C\_COMPILER} --	Путь к компилятору C \\
\texttt{CMAKE\_CXX\_COMPILER} --	Путь к компилятору C++\\
\texttt{CMAKE\_SYSROOT} --	Системный корень (аналог --sysroot в GCC).\\
\texttt{CMAKE\_C\_FLAGS}\\
\texttt{CMAKE\_CXX\_FLAGS}

\section{Использование пакетных менеджеров}

\section{conan}

\subsection*{Простой проект}
\texttt{conan install}\\
\texttt{conan list}\\
\texttt{conanfile.txt}\\
\subsection*{Профили}
\texttt{conan profile detect}

\end{document}
