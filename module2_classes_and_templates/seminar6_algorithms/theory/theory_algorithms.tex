\documentclass{article}
\usepackage[english,russian]{babel}
\usepackage{textcomp}
\usepackage{geometry}
  \geometry{left=2cm}
  \geometry{right=1.5cm}
  \geometry{top=1.5cm}
  \geometry{bottom=2cm}
\usepackage{tikz}
\usepackage{multicol}
\usepackage{listings}
\pagenumbering{gobble}

\lstdefinestyle{csMiptCppStyle}{
  language=C++,
  basicstyle=\linespread{1.1}\ttfamily,
  columns=fixed,
  fontadjust=true,
  basewidth=0.5em,
  keywordstyle=\color{blue}\bfseries,
  commentstyle=\color{gray},
  texcl=true,
  stringstyle=\ttfamily\color{orange!50!black},
  showstringspaces=false,
  %numbers=false,
  numbersep=5pt,
  numberstyle=\tiny\color{black},
  numberfirstline=true,
  stepnumber=1,      
  numbersep=10pt,
  backgroundcolor=\color{white},
  showstringspaces=false,
  captionpos=b,
  breaklines=true
  breakatwhitespace=true,
  xleftmargin=.2in,
  extendedchars=\true,
  keepspaces = true,
  tabsize=4,
  upquote=true,
}
\lstdefinestyle{csMiptCppBorderStyle}{
  style=csMiptCppStyle,
  framexleftmargin=5mm, 
  frame=shadowbox, 
  rulesepcolor=\color{gray}
}

\lstset{style=csMiptCppStyle}
\lstset{literate={~}{{\raisebox{0.5ex}{\texttildelow}}}{1}}


\begin{document}
\title{Семинар \#6: Итераторы и алгоритмы. \vspace{-5ex}}\date{}\maketitle



\section*{Алгоритмы}

\subsection*{Библиотека \texttt{algorithm}}
Библиотека \texttt{algorithm} предоставляет шаблонные функции, которые реализуют множество различных алгоритмов для работы с набором элементов, хранящимся в контейнере. Как правило все эти функции работаю с диапазоном элементов, который задаётся двумя итераторами: итератором на первый элемент диапазона и итератором на элемент, следующим за последним элементом диапазона. 
\subsection*{Функции \texttt{min\_element} и \texttt{max\_element}}
Шаблонные функции \texttt{min\_element} и \texttt{max\_element} находят минимальный и максимальный элемент на некотором диапазоне, задаваемом итераторами. Функции возвращают итератор на этот элемент.
\begin{lstlisting}
#include <iostream>
#include <vector>
#include <string>
#include <list>
#include <algorithm>

int main()
{
	std::vector<int> v {40, 20, 50, 30, 10};
	std::cout << *std::min_element(v.begin(), v.end()) << std::endl;        // 10
	std::cout << *std::min_element(v.begin(), v.begin() + 3) << std::endl;  // 20
	
	std::list<std::string> l {"Tiger", "Lion", "Axolotl"};
	std::cout << *std::min_element(l.begin(), l.end()) << std::endl;        // Axolotl
}
\end{lstlisting}
При этом элемент контейнера должен иметь оператор \texttt{<} для сравнения объектами того же типа. Если такого оператора у типа контейнера нет, то необходимо передать компаратор третьим аргументом в функции. Если этого не сделать, то произойдёт ошибка компиляции.

\subsection*{Функция \texttt{sort}}
Шаблонная функция \texttt{std::sort} сортирует диапазон элементов контейнера. Диапазон задаётся двумя итераторами: итератором на первый элемент и итератором на элемент, следующим за последним элементом.
\begin{lstlisting}
#include <iostream>
#include <vector>
#include <algorithm>

int main()
{
	std::vector<int> v {40, 20, 50, 30, 10};
	std::sort(v.begin(), v.end());
	
	for (auto elem : v)
		std::cout << elem << " ";
	std::cout << std::endl;
}
\end{lstlisting}
Также как и для функций поиска минимума и максимума, элемент контейнера должен иметь оператор \texttt{<} для сравнения объектами того же типа. Ведь для того, чтобы отсортировать элементы, нужно уметь их сравнивать.

Но, в отличии от функций поиска минимума и максимума, функция \texttt{std::sort} налагает более строгие ограничения на итераторы. Дело в том, что \texttt{std::sort} реализована с помощью модернизированого алгоритма быстрой сортировки и при реализации такого алгоритма требуется доступаться к элементу диапазона по его индексу. Функции \texttt{std::sort} передаются на вход два итератора и, чтобы доступаться к элементу диапазона по его индексу,  используется операция прибавления целого числа к итератору. Если у итератора нет такой операции, то функция сортировки работать не будет, произойдёт ошибка компиляции. Например, у итератора \texttt{std::list} нет операции сложения с целым числом, поэтому попытка отсортировать \texttt{std::list} с помощью \texttt{std::sort} приведёт к ошибке.
\begin{lstlisting}
#include <iostream>
#include <list>
#include <algorithm>

int main()
{
	std::list<int> a {40, 20, 50, 30, 10};
	std::sort(a.begin(), a.end());  // Ошибка, нельзя сортировать std::list с помощью std::sort
	
	for (auto elem : a)
		std::cout << elem << " ";
	std::cout << std::endl;
}
\end{lstlisting}
Этот пример показывает, что разные шаблонные функции могут накладывать разные ограничения на объекты с которыми они работают. При работе с такими функциями нужно следить за тем выполняются ли все условия, накладываемые этой шаблонной функцией.

\subsection*{Функция \texttt{reverse}}
Шаблонная функция \texttt{std::reverse} обращает диапазон элементов.
\begin{lstlisting}
#include <iostream>
#include <vector>
#include <string>
#include <list>
#include <algorithm>

int main()
{
	std::vector<int> a {10, 20, 30, 40, 50};
	std::reverse(a.begin(), a.end());
	
	for (auto elem : a)
		std::cout << elem << " ";  // Напечатает 50 40 30 20 10
	std::cout << std::endl;
	
	
	std::list<std::string> b {"Axolotl", "Bat", "Cat"};
	std::reverse(b.begin(), b.end());
	
	for (auto elem : b)
		std::cout << elem << " ";  // Напечатает Cat Bat Axolotl
	std::cout << std::endl;
}
\end{lstlisting}

\subsection*{Функция \texttt{copy}}
Шаблонная функция \texttt{std::copy} принимает на вход три итератора. Первые два итератора задают множество элементов которые нужно скопировать, а третий -- место куда нужно их скопировать. Причём типы первых двух итераторов должны совпадать, а тип третьего итератора может отличаться. То есть можно, например, скопировать элементы вектора в связный список или в другой контейнер. 
\begin{lstlisting}
#include <iostream>
#include <vector>
#include <list>
#include <algorithm>

int main()
{
	std::vector<int> a {10, 20, 30, 40, 50};   // Вектор из элементов  10 20 30 40 50
	std::vector<int> b(5);                     // Вектор из элементов  0 0 0 0 0
 											   //
	std::copy(a.begin(), a.end(), b.begin());  //
	for (auto elem : b)                        //
		std::cout << elem << " ";              // Напечатает 10 20 30 40 50
	std::cout << std::endl;                    //
	                                           //
	std::list<int> c(5);                       // Связный список из элементов 0 0 0 0 0
	std::copy(a.begin(), a.end(), c.begin());  //
	for (auto elem : c)                        //
		std::cout << elem << " ";              // Напечатает 10 20 30 40 50
	std::cout << std::endl;                    //
}
\end{lstlisting}
При работе с функцией \texttt{std::copy} важно помнить, что эта функция сама по себе не добавляет элементы в контейнер, она просто изменяет уже существующие элементы. В том месте куда элементы копируются должно быть достаточно места под копируемые элементы. Если необходимого места не будет, то это неопределённое поведение.
\begin{lstlisting}
#include <iostream>
#include <vector>
#include <algorithm>

int main()
{
	std::vector<int> a {10, 20, 30, 40, 50};   // Вектор из элементов  10 20 30 40 50
	std::vector<int> b;                        // Пустой вектор
	
	std::copy(a.begin(), a.end(), b.begin());  // В векторе b нет места под 5 элементов
	                                           // Это ошибка - неопределённое поведение
}
\end{lstlisting}


\subsection*{Функция \texttt{count}}
Шаблонная функция \texttt{std::count} подсчитывает количество элементов в диапазоне, равных данному.
\begin{lstlisting}
std::vector<int> a {50, 10, 20, 10, 20, 40, 10, 10, 30};
std::cout << std::count(a.begin(), a.end(), 10) << std::endl;    // Напечатает 4
std::cout << std::count(a.begin(), a.end(), 20) << std::endl;    // Напечатает 2
\end{lstlisting}


\subsection*{Функция \texttt{find}}
Шаблонная функция \texttt{std::find} ищет первый элемент в диапазоне элементов, который равен данному. Если такого элемента в диапазоне нет, то функция возврашает итератор на элемент, следующий за последним элементом диапазона.
\begin{lstlisting}
#include <iostream>
#include <vector>
#include <algorithm>

int main()
{
	std::vector<int> a {50, 10, 20, 10, 20, 40, 10, 10, 30};
	
	auto it1 = std::find(a.begin(), a.end(), 40);
	std::cout << (it1 - a.begin()) << std::endl;  // Напечатает 5
		
    auto it2 = std::find(a.begin(), a.end(), 20);
	std::cout << (it2 - a.begin()) << std::endl;  // Напечатает 2
	
	auto it3 = std::find(a.begin(), a.end(), 70);
	if (it3 == a.end())
		std::cout << "Element 70 not found" << std::endl;

	auto it4 = std::find(a.begin(), a.begin() + 3, 40);
	if (it4 == a.begin() + 3)
		std::cout << "Element 40 is not among the first three elements" << std::endl;
}
\end{lstlisting}


\subsection*{Функция \texttt{fill}}
Шаблонная функция \texttt{std::fill} задаёт все элементы диапазона данным значением.
\begin{lstlisting}
#include <iostream>
#include <vector>
#include <list>
#include <string>
#include <algorithm>

int main()
{
	std::vector<int> a {10, 20, 30, 40, 50}
	
	std::fill(a.begin(), a.end(), 70);
	
	for (auto elem : a)                        
		std::cout << elem << " ";  // Напечатает 70 70 70 70 70
	std::cout << std::endl;
}
\end{lstlisting}

\newpage
\subsection*{Функция \texttt{remove} и метод \texttt{erase}}
Шаблонная функция \texttt{std::remove} принимает на вход диапазон элементов, задаваемый двумя итераторами и некоторый элемент. Перемещает все элементы, неравные данному в начало диапазона. Остальные элементы диапазона могут получить произвольные значения. В результате диапазон как бы разделится на две части: первая часть состоит из элементов не равных данному, а вторая часть состоит из произвольных элементов. Функция возвращает итератор на первый элемент второй части. Порядок элементов в первой части диапазон будет тем же, каким он был в изначальном диапазоне.
\begin{lstlisting}
#include <iostream>
#include <vector>
#include <algorithm>

int main()
{
    std::vector<int> a {50, 10, 20, 10, 20, 40, 10, 10, 30};
    auto border = std::remove(a.begin(), a.end(), 10);
    
    for (auto it = a.begin(); it != a.end(); ++it)    
    	std::cout << *it << " ";                      // Напечатает  50 20 20 40 30 40 10 10 30
    std::cout << std::endl;
    
    for (auto it = a.begin(); it != border; ++it)       
    	std::cout << *it << " ";                      // Напечатает  50 20 20 40 30
    std::cout << std::endl;
}
\end{lstlisting}
Шаблонная функция \texttt{std::remove} сама по себе не удаляет элементы, она просто перемещает элементы, не равные данному, в начало диапазона. Чтобы действительно удалить элементы из контейнера нужно воспользоваться методами контейнера. У разных контейнеров способ удаления может различаться, а у некоторых контейнеров (например, у \texttt{std::array}) удалить элементы вообще нельзя. Однако, большинство контейнеров поддерживают метод \texttt{erase}, который принимает два итератора и удаляет все элементы в диапазоне, задаваемом этими итераторами.

\begin{lstlisting}
#include <iostream>
#include <vector>
#include <algorithm>

int main()
{
    std::vector<int> a {50, 10, 20, 10, 20, 40, 10, 10, 30};
    auto border = std::remove(a.begin(), a.end(), 10);
    a.erase(border, a.end());
    
    for (auto it = a.begin(); it != a.end(); ++it)    
    	std::cout << *it << " ";                      // Напечатает  50 20 20 40 30
    std::cout << std::endl;
}
\end{lstlisting}

\newpage
\subsection*{Функция \texttt{unique}}
Шаблонная функция \texttt{std::unique} "удаляет" повтораяющиеся элементы в диапазоне. Работает по тому же принципу, что и функция \texttt{std::remove}. То есть, вместо удаления элементов, перемещает все нужные элементы в начало диапазона, а в оставшейся часть диапазона будут находиться произвольные значения. Возвращает итератор на первый элемент второй части диапазона.
\begin{lstlisting}
#include <iostream>
#include <vector>
#include <algorithm>

int main()
{
    std::vector<int> a {20, 20, 10, 10, 10, 10, 20, 20, 30};
    auto border = std::unique(a.begin(), a.end());
    a.erase(border, a.end());
    
    for (auto it = a.begin(); it != a.end(); ++it)    
    	std::cout << *it << " ";                      // Напечатает  20 10 20 30
    std::cout << std::endl;
}
\end{lstlisting}

\subsection*{Библиотека \texttt{numeric}}
\subsection*{Функция \texttt{iota}}
\subsection*{Функция \texttt{accumulate}}


\newpage

\section*{Input и Output итераторы}
\subsection*{Итератор \texttt{std::back\_insert\_iterator}}
\texttt{std::back\_insert\_iterator<Container>} -- это специальный итератор у которого операторы перегружены по другому. Для него:
\begin{itemize}
\item[--] \texttt{operator*} ничего не делает
\item[--] \texttt{operator++} ничего не делает
\item[--] \texttt{operator=} вызывает метод \texttt{push\_back} контейнера
\end{itemize}

Благодаря таким перегрузкам поведение этого итератора сильно отличается от поведения обычных итераторов. К примеру, следующий код добавит в контейнер \texttt{a} ещё один элемент:
\begin{lstlisting}
std::vector<int> a { 1, 2, 3 };
std::back_insert_iterator<std::vector<int>> it{a};
*it = 4;
\end{lstlisting}

Так как тип этого итератора может иметь длинное название, то была введена функция под названием \texttt{std::back\_inserter}, которая принимает на вход контейнер и возвращает такой итератор. Пример в котором вектор \texttt{a} копируется в пустой вектор \texttt{b} (полная версия в \texttt{std\_copy\_back\_inserter.cpp}):
\begin{lstlisting}
std::vector<int> a {10, 20, 30, 40, 50};
std::vector<int> b;
std::cout << b;

std::copy(a.begin(), a.end(), std::back_inserter(b));
std::cout << b;
\end{lstlisting}

\begin{itemize}
\item Напишите функцию \texttt{append}, которая будет принимать 2 вектора. Первый вектор эта функция должна принимать по ссылке, а второй -- по константной ссылке. Функция должна копировать всё содержимое второго вектора в первый с помощью функции \texttt{std::copy}.
\end{itemize}

\subsection*{Итератор \texttt{std::ostream\_iterator}}
\texttt{std::ostream\_iterator<T>} -- это специальный итератор у которого операторы перегружены по другому:
\begin{itemize}
\item[--] \texttt{operator*} ничего не делает
\item[--] \texttt{operator++} ничего не делает
\item[--] \texttt{operator=} выводит соответствующий элемент в выходной поток (например, \texttt{std::cout} или файл) с помощью оператора \texttt{<{}<} 
\end{itemize}

\begin{itemize}
\item Что сделает эта программа:
\begin{lstlisting}
int main() {
    std::ostream_iterator<int> it{ std::cout, ", " };
    it = 1;
    *it = 2;
    it++ = 3;
}
\end{lstlisting}
\item Напишите программу, которая печатает числа от 1 до 100, разделённые символом \texttt{+}. Используйте \texttt{ostream\_iterator}.
\item Напишите программу, которая печатает содержимое вектора на экран, используя \texttt{std::copy} и \texttt{ostream\_iterator}.
\item Пусть есть такое множество строк:
\begin{lstlisting}
std::set<std::string> a{"Cat", "Dog", "Mouse", "Elephant"};
\end{lstlisting}
Напечатайте содержимое этого множество на экран, каждый элемент на новой строке.
\end{itemize}



\newpage
\section*{Категории итераторов}

\newpage
\section*{Функциональные объекты}
\subsection*{Указатели на функции}
\subsection*{Функторы}
\subsection*{Функторы в стандартной библиотеке}
\subsection*{Лямбда-функции}
\subsection*{\texttt{std::function}}
\subsection*{Указатели на методы, функция \texttt{std::mem\_fn}}


\newpage
\section*{Стандартные алгоритмы с функциональными объектами}


\subsection*{Функция \texttt{for\_each}}
\subsection*{Функция \texttt{find\_if}}
\subsection*{Функция \texttt{count\_if}}
\subsection*{Функции \texttt{all\_of}, \texttt{any\_of} и \texttt{none\_of}}
\subsection*{Функция \texttt{generate}}
\subsection*{Функция \texttt{copy\_if}}
\subsection*{Функция \texttt{transform}}
\subsection*{Функция \texttt{sort}}
\subsection*{Функция \texttt{partition}}
\subsection*{Функция \texttt{stable\_partition}}



\newpage
\section*{Замыкания}



\newpage
.
\newpage
.
\newpage



\section*{Контейнеры}
Стандартная библиотека включает в себя множество разных шаблонных контейнеров и алгоритмов для работы с ними.

\begin{center}
\begin{tabular}{ l | l }
 контейнер & описание и основные свойства \\ \hline


 \texttt{std::vector} & Динамический массив \\
                      & Все элементы лежат вплотную друг к другу, как в массиве \\
                      & Есть доступ по индексу за $O(1)$ \\ \\ \hline
 \texttt{std::list} & Двусвязный список \\
                    & Вставка/удаление элементов за $O(1)$ если есть итератор на элемент \\ \\ \hline
 \texttt{std::forward\_list} & Односвязный список \\
                     & Вставка/удаление элементов за $O(1)$ если есть итератор на предыдущий элемент\\ \\ \hline
 \texttt{std::set} & Реализация множества на основе сбалансированного дерева поиска \\
				   & Хранит элементы без дубликатов, в отсортированном виде\\
				   & Тип элементов должен реализовать \texttt{operator<} (или предоставить компаратор)\\
                   & Поиск/вставка/удаление элементов за $O(\log(N))$ \\ \\ \hline
 \texttt{std::map} & Реализация словаря на основе сбалансированного дерева поиска \\
				   & Хранит пары ключ-значения без дубликатов ключей, в отсортированном виде\\
				   & Тип ключей должен реализовать \texttt{operator<}  (или предоставить компаратор)\\
                   & Поиск/вставка/удаление элементов за $O(\log(N))$ \\ \\ \hline
 \texttt{std::unordered\_set} & Реализация множества на основе хеш-таблицы \\
				   & Хранит элементы без дубликатов, в произвольном порядке\\
                   & Поиск/вставка/удаление элементов за $O(1)$ в среднем \\ \\ \hline
 \texttt{std::unordered\_map} & Реализация словаря на основе хеш-таблицы \\
				   & Хранит пары ключ-значения без дубликатов ключей,в произвольном порядке\\
                   & Поиск/вставка/удаление элементов за $O(1)$ в среднем  \\ \\ \hline
 \texttt{std::multiset} & То же самое, что \texttt{std::set}, но может хранить дублированные значения \\ \\ \hline
 \texttt{std::deque} & Двухсторонняя очередь \\
				     & Добавление/удаление в начало и конец за $O(1)$\\ \\  \hline
 \texttt{std::stack} & Стек \\
 \texttt{std::queue} & Очередь \\
 \texttt{std::priority\_queue} & Очередь с приоритетом \\ \\ \hline
 
 \texttt{std::pair} &  Пара элементов, могут быть объектами разных типов \\
                      & Элементы пары хранятся в публичных полях \texttt{first} и \texttt{second} \\ \\ \hline
 \texttt{std::tuple} &  Фиксированное количество элементов, могут быть объектами разных типов \\ \\ \hline
 \texttt{std::array} &  Массив фиксированного размера, все элементы имеют один тип \\ \\ \hline
\end{tabular}
\end{center}


\newpage
\section*{Итераторы}
Итератор -- это абстракция для итерирования по контейнеру. Многие контейнеры STL имеют вложенный тип \texttt{iterator}. Объекты этого типа используются для итерирования по контейнеру. Контейнеры имеют следующие методы:
\begin{itemize}
\item[--] \texttt{begin()} -- возвращает итератор на первый элемент
\item[--] \texttt{end()}  -- возвращает итератор на фиктивный элемент, следующий после последнего
\end{itemize}
С итератором можно проводить следующие операции:
\begin{itemize}
\item[] \texttt{*it}: перегруженный \texttt{operator*} -- возвращает ссылку на элемент, на который указывает итератор
\item[] \texttt{it++}: переходим к следующему элементу
\item[] \texttt{it1 == it2} и \texttt{it1 != it2}:  операторы равенства и неравенства
\item[] \texttt{it-{}-}: переходим к предыдущему элементу (работает не для всех итераторов)
\item[] \texttt{it + n}: только для итераторов \texttt{vector} -- переходим к элементу, следующему через \texttt{n} элементов после текущего.
\item[] \texttt{it1 - it2}: только для итераторов \texttt{vector} -- возвращает расстояние между элементами
\end{itemize}

Рассмотрим пример работы с итераторами. Приведённый ниже код создаёт массив (\texttt{vector}) и множество на основе бинарного дерева, заполняет их элементами и печатает. Если для вектора поведение итератора очень похоже на указатель, то для множества оно сильно отличается. Например, перегруженный оператор \texttt{++} для итератора \texttt{set} -- это нетривиальная операция перехода к следующему элементу дерева.
\begin{lstlisting}
#include <iostream>
#include <vector>
#include <set>
using namespace std;

int main () {
    vector<int> v = {54, 62, 12, 97, 41, 6, 73};
    for (vector<int>::iterator it = v.begin(); it != v.end(); ++it) {
        cout << *it << " ";
    }
    cout << endl;
    
    set<int> s = {54, 62, 12, 97, 41, 6, 73};
    for (set<int>::iterator it = s.begin(); it != s.end(); ++it) {
        cout << *it << " ";
    }
    cout << endl;
}
\end{lstlisting}

\subsection*{Задачи}
\begin{itemize}
\item Напечатайте только чётные элементы вектора, используйте итераторы
\item Напечатайте каждый второй элемент вектора, используйте итераторы
\item Напишите функцию \texttt{inc}, которая будет принимать на вход вектор целых чисел типа \texttt{int} и будет увеличивать все элементы на \texttt{1}. Для прохода по вектору используйте итераторы.
\item Напечатать содержимое вектора в обратном порядке
\end{itemize}


\newpage
\section*{\texttt{std::vector} и алгоритмы}
В библиотеке \texttt{algorithm} содержится множество алгоритмов, предназначенных для работы с контейнерами STL.
\subsection*{\texttt{std::max\_element} и \texttt{std::min\_element}}
Принимает на вход 2 итератора и возвращает итератор на максимальный элемент на подмассиве, задаваемом этими итераторами. Если таких элементов несколько, то возвращает итератор на первый из них.
\subsection*{Задачи:}
\begin{itemize}
\item На вход подаётся $n$ чисел. Напечатайте минимальный элемент и его индекс.
\begin{center}
\begin{tabular}{ l | l }
 вход & выход \\ \hline
 \texttt{7} & \texttt{1 4}  \\ 
 \texttt{8 2 5 4 1 6 4} &  \\ \hline
 \texttt{1} & \texttt{1 0}  \\ 
 \texttt{1} &  \\
\end{tabular}
\end{center}


\item На вход подаётся чётное количество чисел. Напечатайте минимальный элемент на первой половине и  максимальный элемент второй половины.
\begin{center}
\begin{tabular}{ l | l }
 вход & выход \\ \hline
 \texttt{8} & \texttt{2 9}  \\ 
 \texttt{7 2 8 4 1 9 4 2} &  \\ \hline
 \texttt{8} & \texttt{5 4}  \\ 
 \texttt{8 7 6 5 4 3 2 1} &   \\ \hline
 \texttt{2} & \texttt{5 1}  \\ 
 \texttt{5 1} &   \\
\end{tabular}
\end{center}


\item На вход подаётся $n$ чисел. Напечатайте максимальный элемент, который находится до минимального. Предполагается, что минимальный элемент не является первым.
\begin{center}
\begin{tabular}{ l | l }
 вход & выход \\ \hline
 \texttt{7} & \texttt{8}  \\ 
 \texttt{7 2 8 4 1 9 4} &  \\ \hline
 \texttt{7} & \texttt{2}  \\ 
 \texttt{2 1 2 3 4 5 6} &  \\ \hline
 \texttt{2} & \texttt{3}  \\ 
 \texttt{3 1} &  \\ \hline
\end{tabular}
\end{center}
\end{itemize}

\subsection*{\texttt{std::find}}
Принимает на вход 2 итератора и элемент того же типа, что и тип элементов вектора. Ищет этот элемент и возвращает итератор на этот элемент. Если этого итератора в контейнере нет, то возвращает итератор \texttt{end()} этого контейнера.

\begin{itemize}
\item На вход подаётся $n$ чисел и ещё некоторое число. Напечатайте индекс этого числа в массиве. Если такого числа в массиве нет, то напечатайте \texttt{No such element}.
\begin{center}
\begin{tabular}{ l | l }
 вход & выход \\ \hline
 \texttt{7} & \texttt{5}  \\ 
 \texttt{8 2 5 4 1 6 4} &  \\
 \texttt{6} &  \\ \hline
 \texttt{2} & \texttt{No such element}  \\ 
 \texttt{4 1} &  \\
 \texttt{5} &  \\
\end{tabular}
\end{center}
\end{itemize}

\subsection*{\texttt{std::sort}}
Принимает на вход 2 итератора. Сортирует подмассив, задаваемый этими итераторами, по возрастанию. Сортировка работает за $O(n \log(n))$.

\begin{itemize}
\item На вход подаётся $n$ чисел. Отсортируйте их и напечатайте.
\begin{center}
\begin{tabular}{ l | l }
 вход & выход \\ \hline
 \texttt{8} & \texttt{1 2 4 4 5 6 8 9}  \\ 
 \texttt{8 2 5 4 9 1 6 4} &  \\
\end{tabular}
\end{center}

\item На вход подаётся $n$ строк. Отсортируйте их лексиграфически и напечатайте.
\begin{center}
\begin{tabular}{ l | l }
 вход & выход \\ \hline
 \texttt{5} & \texttt{Cat Cattle Dog Dolphin Elephant}  \\ 
 \texttt{Cat Dog Elephant Cattle Dolphin } &  \\
\end{tabular}
\end{center}
\end{itemize}

\subsection*{\texttt{std::reverse}}
Принимает на вход 2 итератора. Обращает подмассив, задаваемый этими итераторами.
\begin{itemize}
\item На вход подаётся $n$ чисел. Найдите максимум и отсортируйте элементы, идущие до максимума по возрастанию, а все элементы, идущие после максимума -- по убыванию.
\begin{center}
\begin{tabular}{ l | l }
 вход & выход \\ \hline
 \texttt{8} & \texttt{2 4 5 8 9 6 4 1}  \\ 
 \texttt{8 2 5 4 9 1 6 4} &  \\
\end{tabular}
\end{center}
\end{itemize}

\subsection*{\texttt{std::count}}
Принимает на вход 2 итератора и некоторое значение. Находит сколько элементов массива равны этому значению.
\begin{itemize}
\item На вход подаётся $n$ чисел. Найдите сколько элементов массива равны максимальному.
\begin{center}
\begin{tabular}{ l | l }
 вход & выход \\ \hline
 \texttt{8} & \texttt{3}  \\ 
 \texttt{8 2 5 8 8 1 6 4} &  \\
\end{tabular}
\end{center}
\end{itemize}

\subsection*{\texttt{std::accumulate} (библиотека \texttt{numeric})}
Принимает на вход 2 итератора и некоторый объект (начальное значение). Прибавляет все элементы из подмассива, задаваемого итераторами, к начальному значению. В итоге возвращает получившееся значение.
\begin{itemize}
\item На вход подаётся $n$ чисел. Напечатайте сумму этих чисел.
\begin{center}
\begin{tabular}{ l | l }
 вход & выход \\ \hline
 \texttt{8} & \texttt{39}  \\ 
 \texttt{8 2 5 4 9 1 6 4} &  \\ \hline
 \texttt{3} & \texttt{5000000000}  \\ 
 \texttt{2000000000 1000000000 2000000000 } &  \\
\end{tabular}
\end{center}

\item На вход подаётся $n$ чисел и ещё целое число $k$. Напечатайте сумму $k$ наименьших чисел.
\begin{center}
\begin{tabular}{ l | l }
 вход & выход \\ \hline
 \texttt{8} & \texttt{7}  \\ 
 \texttt{8 2 5 4 9 1 6 4} &  \\
 \texttt{3} &\\
\end{tabular}
\end{center}
\end{itemize}

\newpage
\section*{\texttt{std::pair}}
Пара -- это простейший контейнер, который может хранить в себе 2 элемента (возможно, разных типов). Реализация пары имеет примерно следующий вид:
\begin{lstlisting}
template <typename T1, typename T2> struct pair {
    T1 first;
    T2 second;
};
\end{lstlisting}
Для пары определены операторы сравнения. Сравнение происходит в лексиграфическом порядке. То есть для оператора больше сначала сравниваются первые элементы и только если они равны, сравниваются вторые.\\

Для простого создания пар есть шаблонная функция \texttt{make\_pair}. Пару можно создать так:
\begin{lstlisting}
std::pair<string, int> p1 = make_pair("Titanic", 8.4);
std::pair<string, int> p2 {"Titanic", 8.4};
\end{lstlisting}

\begin{itemize}
\item На вход подаётся $n$ чисел. Отсортируйте их и напечатайте сами элементы и их старые индексы. 
\begin{center}
\begin{tabular}{ l | l }
 вход & выход \\ \hline
 \texttt{8} &                \texttt{1 2 4 4 5 6 8 9}  \\ 
 \texttt{8 2 5 4 9 1 6 4} &  \texttt{5 1 3 7 2 6 0 4}\\
\end{tabular}
\end{center}

\item На вход подаётся $n$ фильмов. Передаются названия фильмов и их рейтинг на кинопоиске. Отсортировать эти фильмы по возрастанию рейтинга.
\begin{center}
\begin{tabular}{ l | l }
 вход & выход \\ \hline
 \texttt{5}              & \texttt{Venom2 6.2}  \\ 
 \texttt{TheMatrix 8.5}  & \texttt{Shrek 8.0}  \\
 \texttt{Titanic 8.4}    & \texttt{Titanic 8.4}\\
 \texttt{GreenMile 8.9}  & \texttt{TheMatrix 8.5}\\
 \texttt{Shrek 8.0}      & \texttt{GreenMile 8.9}\\
 \texttt{Venom2 6.2}     &\\
\end{tabular}
\end{center}
\end{itemize}
\newpage
\section*{\texttt{std::list}}
\texttt{std::list} -- Это двусвязный список. Основные методы для работы со списком:

\begin{center}
\begin{tabular}{ l | l }
 метод & описание \\ \hline
 \texttt{insert} & Принимает на вход итератор и некоторый объект и вставляет этот объект\\
                 & перед элементом, на который указывает итератор.  \\ \\ \hline
 \texttt{erase}  & Принимает на вход итератор и удаляет элемент.\\
                 & Переданный итератор, конечно, становится недействительным,  \\ 
                 & поэтому этот метод возвращает корректный итератор на элемент, следующий за удалённым \\ \\ \hline
 \texttt{push\_back}  & Добавить элемент в конец.\\
 \texttt{push\_front}  & Добавить элемент в начало.\\
 \texttt{pop\_back}   & Удалить элемент из конца.\\
 \texttt{pop\_front}  & Добавить элемент из начала.\\ \\ \hline
 \texttt{sort}       & Сортирует список (функция \texttt{sort} из библиотеки \texttt{algorithm} для списка не работает) 
\end{tabular}
\end{center}

К итераторам \texttt{std::list<T>::iterator} нельзя прибавлять целые числа, вместо этого нужно использовать функцию \texttt{std::advance}. Также эти итераторы нальзя вычитать, вместо этого нужно использовать функцию \texttt{std::distance}. Эти две функции работают за линейное время.

\begin{itemize}
\item На вход подаётся $n$ чисел. Сохраните эти числа в связном списке, найдите их сумму и напечатайте её.
\item На вход подаётся $n$ чисел. Сохраните эти числа в связном списке, отсортируйте список и напечатайте их.

\item На вход подаётся $n$ чисел. Сохраните эти числа в связном списке. Скопируйте все элементы списка в его конец и напечатайте его.
\begin{center}
\begin{tabular}{ l | l }
 вход & выход \\ \hline
 \texttt{5} & \texttt{8 2 1 4 2 8 2 1 4 2}  \\ 
 \texttt{8 2 1 4 2} &  \\
\end{tabular}
\end{center}

\item Напишите функцию \texttt{insert\_after}, которая будет принимать на вход список из чисел типа \texttt{int}, итератор на элемент этого списка и некоторое число \texttt{x}. Функция доджна вставлять \texttt{x} после элемента, заданным итератором.

\item Проверьте только что написанную функцию. На вход подаётся $n$ чисел. Сохраните эти числа в связном списке. Продублируйте каждый элемент списка.
\begin{center}
\begin{tabular}{ l | l }
 вход & выход \\ \hline
 \texttt{5} & \texttt{8 8 2 2 1 1 4 4 2 2}  \\ 
 \texttt{8 2 1 4 2} &  \\
\end{tabular}
\end{center}

\item На вход подаётся $n$ чисел. Сохраните эти числа в связном списке. Удалите все чётные числа из списка и напечатайте его.
\begin{center}
\begin{tabular}{ l | l }
 вход & выход \\ \hline
 \texttt{5} & \texttt{1 5}  \\ 
 \texttt{8 2 1 4 5} &  \\
\end{tabular}
\end{center}

\end{itemize}


\newpage


\end{document}